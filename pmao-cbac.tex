\documentclass[a4paper,fleqn, review]{cas-dc}

\graphicspath{{Figure/}}

\usepackage{hyperref}
\usepackage{comment}
\usepackage{algorithm, algpseudocode}
\usepackage{amsmath}
%\usepackage{booktabs}
%\usepackage{cleveref}
\usepackage{colortbl}
\usepackage{makecell}
\usepackage{xr}
\externaldocument{supp}

\usepackage[numbers]{natbib}
\usepackage{subfig}
\usepackage{changepage}

\usepackage{tikz,pgfplots,pgfplotstable}
\usetikzlibrary{arrows,calc,positioning}
\usetikzlibrary{shapes,chains}
\pgfplotsset{compat=newest}

%\theoremstyle{thmstyleone}\newtheorem{theorem}{Theorem}\newtheorem{proposition}[theorem]{Proposition}\theoremstyle{thmstyletwo}\newtheorem{example}{Example}\newtheorem{remark}{Remark}\theoremstyle{thmstylethree}\newtheorem{definition}{Definition}
\renewcommand{\algorithmiccomment}[1]{~{\footnotesize\ttfamily{/*#1*/}}}
\newcolumntype{L}[1]{>{\raggedright\let\newline\\\arraybackslash\hspace{0pt}}m{#1}}
\newcolumntype{C}[1]{>{\centering\let\newline\\\arraybackslash\hspace{0pt}}m{#1}}
\newcolumntype{R}[1]{>{\raggedleft\let\newline\\\arraybackslash\hspace{0pt}}m{#1}}

\begin{document}
	
%\let\WriteBookmarks\relax
%\def\floatpagepagefraction{1}
%\def\textpagefraction{.001}
\shorttitle{PASTA with many application-aware optimization criteria for alignment based phylogeny inference}
\shortauthors{Nayeem et~al.}

\title [mode = title]{PASTA with many application-aware optimization criteria for alignment based phylogeny inference}                      


\author[1]{Muhammad Ali Nayeem} 
\author[1]{Md. Shamsuzzoha Bayzid}
\author[1]{Naser Anjum Samudro}
\author[1]{Mohammad Saifur Rahman}
\author[1]{M. Sohel Rahman} \cormark[1]
\address[1]{Department of Computer Science \& Engineering, Bangladesh University of Engineering \& Technology, Dhaka 1205, Bangladesh}

\cortext[cor1]{Corresponding author. \href{msrahman@cse.buet.ac.bd}{Email:msrahman@cse.buet.ac.bd}}


\begin{abstract}
	Multiple sequence alignment (MSA) is a prerequisite for several analyses in bioinformatics such as phylogeny estimation, protein structure prediction, etc. PASTA (Practical Alignments using SAT\'e and TrAnsitivity) is a state-of-the-art method for computing MSAs, well-known for its accuracy and scalability. It iteratively co-estimates both MSA and maximum likelihood (ML) phylogenetic tree. It attempts to exploit the close association between the accuracy of an MSA and the corresponding tree in finding the output through multiple iterations from both directions. Currently, PASTA uses the ML score as its optimization criterion which is a good score in phylogeny estimation but cannot be proven as a necessary and sufficient criterion to produce an accurate phylogenetic tree. Therefore the integration of multiple application-aware objectives, carefully chosen considering better association to the tree accuracy, into PASTA may potentially have a profound positive impact on its performance. This paper employed four application-aware objectives alongside ML score to develop a multi-objective (MO) framework, namely, PMAO, that leverages PASTA to generate a bunch of high-quality solutions that are considered equivalent in the context of conflicting objectives under consideration. We analyzed this tree-space based on the tree generated by PASTA by experimenting on a popular biological benchmark and found that the tree-space contains significantly better trees than PASTA. 
	To help the domain expert further in choosing the most appropriate tree from the PMAO output (containing a relatively large set of high-quality solutions), we added an additional component within the PMAO framework that is capable of generating a smaller set of high-quality solutions. Finally, we attempted to obtain a single high-quality solution without using any external evidence and found that summarizing the few solutions detected through the above component can serve this purpose to some extent.
\end{abstract}

\begin{graphicalabstract}
	\includegraphics[width=1.0\textwidth]{graphical-abs.pdf}
\end{graphicalabstract}

\begin{highlights}
	\item We develop the PMAO framework, based on PASTA, by incorporating many application-aware objective functions through principles of multi-objective optimization, to generate a bunch of high-quality phylogenetic trees.
	\item We innovatively employ supervised machine learning as well as as well as some simple criteria within the PMAO framework to generate a smaller number of top solutions to assist the domain expert in choosing the final solution through visual inspection.
	\item We experiment with summarizing the PMAO outputs using greedy consensus and quartet consistency to obtain a single high-quality solution without using any external evidence.
\end{highlights}

\begin{keywords}
	Multiple sequence alignment \sep Phylogenetic tree \sep Multi-objective optimization
\end{keywords}



\maketitle

\section{Introduction}
\label{sec:intro}
Multiple sequence alignment (MSA) aims to arrange more than two biological sequences such that each site in the resultant alignment holds homologous characters. The gaps placed in an aligned sequence seek to reflect the historical insertion/deletion events as closely as possible. MSA is used as an essential step in several biological studies, such as prediction of structure/function of newly discovered proteins, estimation of phylogeny among a group of species, etc. This paper addresses the MSA task in the context of phylogeny estimation from sequence data, which usually comprises two phases, namely, (A) the computation of an MSA, and subsequently (B) the inference of a tree therefrom. The characteristics of the MSA obtained in Phase A dramatically influences the accuracy of Phase B. Thus an MSA tool that is aware of its intended usage (i.e., phylogeny estimation in our case) is expected to yield output of higher quality~\cite{nayeem2020multiobjective, nayeem2019phylogeny}.  

A huge number of MSA methods are available in the literature. We can broadly divide those into three categories: progressive, consistency-based, and iterative. This division is not exclusive as many tools also use a combination of these techniques. Among them the most flexible are the iterative methods (e.g., SAT\'e~\cite{liu2009rapid}, SAT\'e-II~\cite{liu2012sate}, PASTA~\cite{mirarab2015pasta}). They can fix errors made in the earlier stages of computation by repeating some steps until an optimization criterion or objective function, quantifying the quality of the (re)alignment, converges. Due to such an advantage, progressive (e.g., MUSCLE~\cite{edgar2004muscle}, MAFFT~\cite{katoh2002mafft}, etc.) and consistency-based (e.g., T-COFFEE~\cite{notredame2000t}, ProbCons~\cite{do2005probcons}, etc) methods also employ a iterative refinement phase at the end of their pipelines. Notably, various objectives (e.g., sum-of-pairs measure and its weighted variants, consistency score, etc.) have been used in the literature for iterative improvement of MSAs. 

The efficacy of using several different objective functions to compare candidate MSAs persuaded researchers (\cite{da2010alineaga, ortuno2013optimizing, soto2014multi, abbasi2015local, rubio2016hybrid,zambrano2017comparing, rubio2018characteristic, benitez2020sequoya}) to employ multi-objective (MO) optimization. We were motivated to explore such an approach since the alignment optimized under one objective may be different from the alignments generated under other objectives, inferring discordant homologies relating to the sequences under consideration. MO optimization can address this issue by
optimizing multiple conflicting objectives simultaneously to
generate a set of alternative alignments. Also, as no single objective is biologically guaranteed to lead to the most accurate solution, the argument of combining alternative criteria seems reasonable. However, such an approach needs to be backed by the choice of appropriate objective functions and performance measures that are not addressed in the existing MO literature on MSA~\cite{nayeem2020multiobjective}.  


Traditionally, the MSA methods are benchmarked based on two alignment quality metrics: SP-score and TC-score~\cite{warnow2017computational}. These measures compare the estimated alignment to the reference alignment (i.e., the ground truth). In ~\cite{nayeem2020multiobjective, nayeem2019phylogeny}, the authors argued with experimental evidence that such generic measures might not be adequate to choose the best MSA method to perform a specific biological task (e.g., protein structure prediction, phylogeny estimation, etc.). Instead, they proposed applying a domain-specific measure that can potentially capture to what extent the output can serve the actual purpose. Taking phylogeny estimation as the intended application, they demonstrated the advantages of using tree quality for performance evaluation instead of traditional measures. They developed a systematic method to identify application-aware objective functions based on their correlation to the tree quality. It was subsequently shown, through extensive experiments, that optimizing those objectives by MO techniques can yield high-quality phylogenetic trees.



PASTA is a state-of-the-art MSA method that exhibits better accuracy and scalability than other methods. It iteratively co-estimates both an MSA and the corresponding phylogenetic tree till the maximum likelihood (ML) score of the newly computed (MSA, tree) pair improves. By default, the first iteration constructs an ML tree from an initial alignment as the guide tree. Each iteration consists of six steps as follows. As the 1st step, it decomposes the set of unaligned sequences into disjoint subsets by applying \textit{mincluster} technique~\cite{balaban2019treecluster} on the guide tree. The 2nd step computes a spanning tree on the subsets of sequences. Next each subsets are aligned in the 3rd step to generate \textit{type-1 sub-alignments}. The 4th step aligns each pair of \textit{type-1 sub-alignments} on each edge of the spanning tree obtaining \textit{type-2 sub-alignments} which are merged using transitivity to produce the final MSA in the 5th step. In the 6th step, an ML tree is inferred from the final MSA as the guide tree for the next iteration. PASTA is also termed as a `meta-method' as it leverages other methods (e.g., MAFFT, FastTree-2, OPAL~\cite{wheeler2007multiple}, etc.) in its internal steps. 

PASTA, extended from SAT\'e-II, can be seen as an application-oriented aligner 
as it makes an effort to exploit the close association between the
accuracy of an MSA and the corresponding tree in finding the output through multiple iterations from both directions. 
This feature further motivates us to incorporate more application-aware objectives within the internal steps of PASTA, expecting that this would further improve the efficacy thereof. This paper makes the following contribution in this direction. 

\begin{itemize}
	\item We develop a decomposition-based MO framework, namely, PMAO (PASTA with Many\footnote{ MO literature refers more than three objectives are as \textit{many}\cite{li2015many} due to the added complexities to handle them} Application-aware Objectives), by extending PASTA to incorporate five application-aware objectives. The PMAO framework can lead to a tree-space containing significantly better trees than PASTA. 
	
	\item Due to the inherent nature of MO optimization algorithms employed therein, PMAO outputs a bunch of high-quality trees (i.e., non-dominated Pareto-optimal solutions). As part of the PMAO framework, we develop a machine learning as well as some simple criteria based approach to identify a few solutions containing at least one high-quality tree. This feature can assist the domain expert in choosing the final solution through visual inspection. To the best of our knowledge, this is a unique approach to filter a large set of Pareto-optimal solutions that could be useful in other MO optimization tasks as well. 
	
	\item We further attempted to obtain a single high-quality solution without using any external evidence by summarizing the few solutions detected through above approach using greedy consensus and quartet consistency. The results suggest that summarizing can serve this purpose to some extent.
	
\end{itemize}


\section{Methods}
\label{sec:method}

\begin{figure*}[!htbp]
	\begin{adjustwidth}{-1.1cm}{}
		\centering
		\subfloat[Input-output]{\includegraphics[width=0.28\textwidth]{PMAO}}\subfloat[A high-level workflow for one weight vector]{\label{fig:PMAO:flow}
\colorlet{lcfree}{green}
\colorlet{lcnorm}{blue}
\colorlet{lccong}{red}

\pgfdeclarelayer{marx}
\pgfsetlayers{main,marx}
\providecommand{\cmark}[2][]{\begin{pgfonlayer}{marx}
    \node [nmark] at (c#2#1) {#2};
  \end{pgfonlayer}{marx}
  } 
\providecommand{\cmark}[2][]{\relax} 


\resizebox{0.44\textwidth}{!}{\begin{tikzpicture}[>=triangle 60,              start chain=going below,    node distance=6mm and 60mm, every join/.style={norm},   ]
\tikzset{
  base/.style={draw, on chain, on grid, align=center, minimum height=4ex},
  proc/.style={base, rectangle, text width=19em},
  proc2/.style={base, rectangle, text width=12em},
  test/.style={base, diamond, aspect=2, text width=6em},
  term/.style={proc, rounded corners},
  term2/.style={proc2, rounded corners},
  merge/.style={base, circle},
coord/.style={coordinate, on chain, on grid, node distance=6mm and 25mm},
nmark/.style={draw, cyan, circle, font={\sffamily\bfseries}},
norm/.style={->, draw, lcnorm},
  free/.style={->, draw, lcfree},
  cong/.style={->, draw, lccong},
  it/.style={font={\small\itshape}}
}
\node [term, fill=lcfree!25] {Input: unaligned sequences,\\ \scriptsize{$<W_{SIMG},W_{SIMNG},W_{SOP},W_{GAP},W_{ML}>$}};
\node [proc, join] (p0) {$iA \gets$ compute initial alignment\\$iT \gets$ infer ML tree};
\node [proc, join,  fill=lcfree!25] (p01) {$stocDecom \gets False$};
\node [proc, join, fill=lcfree!25] (p12) {\scriptsize{$SIMG, SIMNG, SOP, GAP, ML \gets$} calculate 5 scores from ($iA, iT$)};
\node [proc, join, , fill=lcfree!25] (p13) {\scriptsize{$bS \gets SIMG \times W_{SIMG} + SIMNG \times W_{SIMNG} + SOP \times W_{SOP} + GAP \times W_{GAP} + ML \times W_{ML}$}};
\node [proc, join] (p14) {($nA, nT) \gets$ run a PASTA iteration to get a new (alignment, tree) pair from ($iA, iT$)\\{~{\footnotesize\ttfamily{/*$stocDecom$  triggers the stochastic decomposition*/}}} };
\node [proc, join, fill=lcfree!25] (p15) {\scriptsize{$SIMG, SIMNG, SOP, GAP, ML \gets$} calculate 5 scores from ($nA, nT$)};
\node [proc, join, fill=lcfree!25] (p16) {\scriptsize{$nS \gets SIMG \times W_{SIMG} + SIMNG \times W_{SIMNG} + SOP \times W_{SOP} + GAP \times W_{GAP} + ML \times W_{ML}$}};
\node [test,right=6.3cm of p0] (t1) {$(nS > bS)$?};
\node [proc2] (p2) {($bA, bT) \gets (nA, nT)$\\$bS \gets nS$};
\node [proc2, join,  fill=lcfree!25] (p24) {$stocDecom \gets False$};
\node [proc2,  fill=lcfree!25] (p23) {$stocDecom \gets True$};
\node [merge, join] (p21) {};
\node [proc2, join] (p22) {($iA, iT) \gets (nA, nT)$};
\node [test, join] (t6) {terminate?}; \node [term2, join] (p10) {Return ($bA, bT$)};
\node [coord, right=2.2cm of t1] (c1)  {}; \node [coord, left=of t6] (c6)  {}; \node [coord, left=of t6, xshift=-17em]  (cS1)  {}; 

\path (t1.south) to node [near start, xshift=1em] {$y$} (p2);
  \draw [*->,lcnorm] (t1.south) -- (p2);
\path (t6.south) to node [near start, xshift=1em] {$y$} (p10); 
  \draw [*->,lcnorm] (t6.south) -- (p10);
\path (t1.east) to node [near start, yshift=1em] {$n$} (c1); 
  \draw [*->,lcnorm] (t1.east) -- (c1) |- (p23);
  
\draw [->,lcnorm] (p24.west) -- ++(-4mm,0) |- (p21);


\path (t6.west) to node [yshift=-1em] {$n$} (c6); 
  \draw [*->,lcnorm] (t6.west) -- ++(-16mm,0)  |- (p14.east);



\draw [->,lcnorm]
  (p16.south) -- ++(5mm,-3mm)  -- ++(31mm,0) 
  |- node [black, near end, yshift=0.75em, it] {} (t1.west);
  
\end{tikzpicture}
}

 }
\subfloat[30 well-spaced 5D weight vectors]{\label{fig:weight}\includegraphics[width=0.35\textwidth]{30-weight}}
	\end{adjustwidth}
	\caption{A simplified illustration of our PMAO framework.}
	\label{fig:PMAO}
\end{figure*}

\subsection{Application-aware objective functions}
Alongside the ML score, we incorporate the following four simple objective functions, identified by~\cite{nayeem2020multiobjective} based on their better correlation to the tree accuracy, in our PMAO framework. Several pairs of these objectives may have conflicting relationship~\cite{nayeem2020multiobjective}.
\begin{enumerate}
	\item Maximize similarity for columns containing gaps (SIMG): For each column of the MSA having at least one gap, it calculates the ratio of the most frequent characters. Then all those ratios are added to get the SIMG score.
	\item Maximize similarity for columns containing no gaps (SIMNG): This is similar to SIMG except that it considers those columns of the MSA that do not have any gap.
	\item Maximize sum-of-pairs (SOP): For each pair of aligned sequences in the MSA, it takes the sum of substitution score for the two aligned characters across all columns using a substitution matrix. The addition of all pairwise scores gives the SOP score. In this paper, we use the BLOSUM62 matrix.
	\item Minimize the number of gaps (GAP): The summation of the number of gap characters in each aligned sequence. For the sake of uniformity, we convert this score into a maximization criterion.
\end{enumerate}

The above objectives can assess the alignment independently of the type of input sequences (i.e., DNA, RNA, protein). Thus the PMAO framework is equally applicable for DNA, RNA, and proteins sequences unlike some methods which are are specialized for protein sequences such as ProbCons, Probalign, etc. The only change needed in this regard is as follows. In this paper, we use BLOSUM62 substitution matrix for computing the SOP objective for protein sequences; we need to specify a suitable matrix (e.g., NUC4.4) to work on DNA or RNA sequences.

\subsection{PMAO framework}
\subsubsection{MO principles}
The goal of an MO algorithm is to generate a set of solutions, popularly known as the Pareto-optimal solutions in the MO literature, which represent the best compromise among the (conflicting) objectives.
Among the several classes of MO algorithms (e.g., pareto-based, decomposition-based, indicator-based, etc.), decomposition-based strategies are found effective to face the difficulties in handling `many' (i.e., more than three) objectives~\cite{li2015many}. These algorithms decompose the task of generating several alternative solutions into many single-objective problems with the help of a set of well-distributed weight vectors, popularly known as reference directions. Each weight vector aggregates the different objective scores into a single value that eventually leads to one member of the final solution set.

\subsubsection{Simplified workflow}
We develop a decomposition-based MO framework, namely, PASTA with many application-aware objectives (PMAO) (Figure~\ref{fig:PMAO}) by driving the search process of PASTA with a total of five objectives directed by a 5D weight vector. Figure~\ref{fig:PMAO:flow} depicts a high-level workflow for one weight vector, where the steps inspired by the MO approach are marked as green. This workflow is executed for all weight vectors to get alternative solutions and can be performed independently in parallel. As will be evident later, PMAO treats a solution better than the other based on the weighted-sum of five objective values instead of using ML score alone. Also, note that PMAO keeps track of whether an improved solution is generated at the previous iteration through a boolean variable \textit{stocDecom}. It impacts the divide-and-conquer strategy within PASTA by enabling the stochastic decomposition, which will be discussed soon. PMAO uses the default behavior of PASTA unless mentioned otherwise.

\subsubsection{Weight vectors}
Although working with a higher number of weight vectors increase the chance of getting better solutions in the solution set, we choose to work with 30 weight vectors to reduce the computational burden as well as to demonstrate the synergy between PASTA and an MO approach since 30 is quite a low number to tackle 5 objectives alone by an MO algorithm~\cite{deb2014evolutionary}. We calculate 30 well-spaced points on a 5D unit simplex using the method suggested by~\cite{ref_dirs_energy} as our weight vectors. Each of the 30 vertical bars in Figure~\ref{fig:weight} depicts one weight vector. 

\begin{algorithm}[!htbp]\scriptsize
	\textbf{Input:} $tree$: to be bisected; $maxSize$: max. allowable leaves in a tree; $stocDecom$: triggers the stochastic decomposition
\begin{algorithmic}[1]
		\caption{min-cluster-size-bisect}
		\label{algo:min-bisect}
		\State{$nodeLeaves \gets$  empty dictionary}
		\For{each node $b$ in post-order-traverse($tree$)}
		\For{each child $c$ of node $b$}
		\State $nodeLeaves[ch] \gets $ \Call{leaf-count}{$ch$}
		\EndFor
		\If{ \Call{leaf-count}{$b$} $> maxSize$}
		\If{$stocDecom = False$}
		\State $selected \gets$ the node $x$ with the maximum $nodeLeaves[x]$ value
		\Else
		\State $selected \gets $ fitness proportionate selection where selection probability of a node $x \propto nodeLeaves[x]$ \Comment{stocastic decompostion} \label{algo:min-bisect:stoc}
		\EndIf
		\EndIf
		\State $t1 \gets $ the subtree of $tree$ rooted at $selected$
		\State remove $selected$ from its parent in $tree$
		\EndFor
		\State \textbf{return} $tree, t1$
		\Statex

		\Function{leaf-count}{$node$, $tree$}
		\State $ count $ $\gets$ no. of leaves in the subtree of $tree$ rooted at $node$
		\State \textbf{return} $ count $
		\EndFunction
	\end{algorithmic}
\end{algorithm}

\subsubsection{Stochastic decomposition}\label{subsec:stocastic}
We also enhance PASTA's divide-and-conquer method in the context of MO principles due to its huge impact on the accuracy of the generated (MSA, tree) pair~\cite{liu2012sate}. The heart of this strategy is a decomposition method that divides the leaves (i.e., unaligned sequences) of the guide tree into disjoint subsets. Since version 1.8.0, PASTA has been using \textit{mincluster} decomposition~\cite{balaban2019treecluster} which minimizes the number of resultant subsets given the maximum allowable members in a subset (\textit{maxSize} parameter in Algorithm~\ref{algo:min-bisect}). The default value of this parameter is set to half of the total leaf count. \textit{mincluster} strategy is implemented by repeatedly calling a method, namely, \textit{min-cluster-size-bisect}, to bisect a given tree. Scanning the nodes of the input tree in a post-order manner, it removes the subtree with the maximum number of leaves not exceeding the \textit{maxSize}. We embed some randomness in this method to (i) help PASTA to escape local optima and to (ii) increase the diversity of the solution set generated by PMAO. Line~\ref{algo:min-bisect:stoc} of Algorithm~\ref{algo:min-bisect} enforce the idea of stochastic decomposition, which randomly picks a subtree with the selection probability proportional to the number of leaves under that subtree.

\begin{algorithm}[!htbp]\scriptsize
	\textbf{Input:} $SIMG, SIMNG, SOP, GAP, ML$: scores to be normalized
\begin{algorithmic}[1]
		\caption{rough-normalization}
		\label{algo:normalize}
		\State $GAP \gets 1.0/GAP$ \Comment{convert into a maximization score}
		\State $ML \gets -1.0/ML$ \Comment{shift the value range to positive zone}
		\State $ obj \gets [SIMG, SIMNG, SOP, GAP, ML]$
		\State $max \gets$ the maximum value in $obj$
		\State $max \gets$ cast $max$ as integer
		\State $d \gets$ the no. of digits in $max$
		\State $ base \gets 10^{d+1}$
		\State add $base$ to all values in $obj$
		\For{$i \gets$ 0 to 4}
		\State $obj[i] \gets $ \Call{softsign}{$obj[i]$}
		\EndFor
		\State \textbf{return} $obj$
		\Statex

		\Function{softsign}{$x$} \label{algo:normalize:softsign}
		\State \textbf{return} $ \frac{x}{1 + |x|} $
		\EndFunction
	\end{algorithmic}
\end{algorithm}

\begin{figure}[!htbp]
	\centering
	\includegraphics[width=0.46\textwidth]{sigmoid}
	\caption{Some sigmoid functions. Image taken from WIKIPEDIA.}
	\label{fig:sigmoid}
\end{figure}

\subsubsection{Normalization}
Each of the five objective functions is measured using different scales, and their ranges differ to a great extent. So without normalization, their weighted-sum may be unexpectedly biased towards the objectives whose range extends far right, on the real number line, to the others. To add further complication, we do not have enough information regarding those objective values' distribution (e.g., min, max, avg, etc.) as we cannnot calculate them in a straightforward way . So we design a \textit{rough-normalization} method outlined in Algorithm~\ref{algo:normalize} which is called before the aggregation. Here we first transform the objective values to have the equal number of integer digits by adding an offset. Then we apply the softsign function (line~\ref{algo:normalize:softsign}) on each transformed value to make them close to 1.0. We choose the softsign function, over other sigmoid functions (Figure~\ref{fig:sigmoid}), as it converges polynomially (rather than exponentially) which helps to reduce further the risk of the aggregation being unjustly biased towards some objectives. 


\subsection{Domain-specific performance measure}
As we consider phylogeny estimation as the application domain of MSA, we assess the performance of PMAO solely based on the ML trees from its solution set. Therefore, we evaluate the quality of each ML tree with respect to the true phylogenetic tree
using a widely used measure known as the False Negative (FN) rate. FN rate is the percentage of edges present in the true tree but missing in the estimated tree. So a smaller value of the FN rate
is desirable. Although there are two more common tree error measures: False Positive rate and Robinson-Foulds rate, all of them are identical when the true and estimated trees are
binary~\cite{warnow2017computational} trees, which is the case in this paper.


\subsection{Datasets}
We conduct our experiments on the BAliBASE 3.0 benchmark~\cite{thompson2005balibase} which is the most widely used alignment database of protein families. It provides manually refined reference alignments of high quality based on 3D structural superposition. It has 218 datasets which are organized into six groups according to their families and similarities: RV11
(very divergent sequences, residue identity below 20\%), RV12 (medium to divergent sequences, 20\%-40\% residue identity), RV20 (families with one or more highly divergent sequences), RV30 (divergent subfamilies), RV40 (sequences with large terminal N/C extensions), and RV50 (sequences with large internal insertions). To compare among different variants of PASTA and PMAO, we randomly sample 51 datasets in a stratified way within six groups in two rounds. We show them under set A and set B in Table~\ref{tab:balibase} for ease of presentation. The only difference between set A and set B is that they are sampled in two different rounds. Moreover, keeping these two sets separate further helps us experiment with our supervised machine learning module where we used set A for training and set B for testing purposes. 
Notably, in this paper we primarily focused on improving the accuracy of PASTA and hence designed our experiments accordingly. We did not experiment scalability by working on larger datasets as PASTA is considered as a highly scalable tool.

\begin{table}[!htbp]
	\scriptsize
\caption{ Datasets selected randomly from BAliBASE 3.0 benchmark.}
	\begin{adjustwidth}{-0.0cm}{}
	\begin{tabular}{l|L{3.6cm}|L{2.5cm}}
		%\hline
		Group & Set A & Set B \\
		\hline
		RV11  & BB11005, BB11018, BB11020, BB11033 & BB11007, BB11019, BB11034, BB11038 \\
		\hline
		RV12  & BB12001, BB12013, BB12022, BB12035, BB12044 & BB12005, BB12026, BB12029,  BB12037\\
		\hline
		RV20  & BB20001, BB20010, BB20022, BB20033, BB20041 & BB20002, BB20012, BB20030, BB20037\\
		\hline
		RV30  & BB30002, BB30008, BB30015, BB30022 & BB30003 BB30011, BB30021, BB30026\\
		\hline
		RV40  & BB40001, BB40013, BB40025, BB40038, BB40048 & BB40006, BB40009, BB40019, BB40033 \\ \hline
		RV50  & BB50001, BB50005, BB50010, BB50016 &  BB50006 BB50002, BB50009, BB50014 \\
		%\hline
	\end{tabular}\label{tab:balibase}\end{adjustwidth}
\end{table}

As we adopt the FN rate as our domain-specific performance measure, following the strategy of~\cite{mirarab2015pasta} we generate a reference tree for each dataset by inferring an ML tree from the reference alignments using RAxML~\cite{stamatakis2014raxml} with bootstrapping and retaining only the highly supported edges.

 \section{Results and discussion}
\label{sec:experiment}

\subsection{Impact of stochastic decomposition}
To assess the impact of the underlying stochastic decomposition method on PMAO, we create four variants of PMAO and PASTA (Table~\ref{tab:variants}). In particular, we vary the number of iterations and consider both \textit{mincluster} based (i.e., the default for PASTA) and stochastic decomposition. Increasing the iteration potentially intensifies the effect of the stochastic decomposition in PMAO framework. In Tables~\ref{tab:pmao-variants-a},  \ref{tab:pmao-variants-b}, we report the best FN rates among the 30 solutions generated by PMAO variants for datasets under set A and B, respectively; lower (i.e., better) FN rate values are marked with a darker shade.
We observe that the PMAO-3I-S and PMAO-8I-S columns contain more dark shades than PMAO-3I-D and PMAO-8I-D columns, which indicates the positive impact of stochastic decomposition on the performance of PMAO framework. To contrast among these variants, we apply the Friedman Aligned Ranks test~\cite{hodges2012rank} followed by complementary Holm’s post-hoc procedure~\cite{holm1979simple} on Tables~\ref{tab:pmao-variants-a},  \ref{tab:pmao-variants-b}, as recommended by~\cite{derrac2011practical, rodriguez-fdez2015stac} using 95\% confidence level. The results the summarized in Table~\ref{tab:test-pmao-variants}. The lower ranks of stochastic variants and significant difference between PMAO-3I-D and PMAO-8I-S suggest the effectiveness of stochastic decomposition. Note that PASTA uses three iterations by default, and most of its improvement is achieved in the first iteration. So, increasing iteration is not expected to improve PASTA output significantly. Moreover, stochastic decomposition makes sense in the context of MO principles as discussed earlier in Section~\ref{subsec:stocastic}. We have verified this by conducting similar analyses (Supplementary Tables~\ref{tab:pasta-variants-a}, \ref{tab:pasta-variants-b},  \ref{tab:test-pasta-variants}) based on four PASTA variants. The PASTA-3I-S variant does not get ample opportunity to leverage the stochastic decomposition due to its single execution.
In contrast, the PMAO-3I-S variant executes that 30 times on different search spaces defined by the weight vectors. We find no significant difference between any pair of PASTA variants as expected (Supplementary Table~\ref{tab:test-pasta-variants}), and PASTA-8I-D seems to be the best variant (more dark shades). In the subsequent sections, we use the PMAO-8I-S and the PASTA-8I-D variants for a fair comparison.


\begin{table}[!htbp]
	\small
  \centering
  \caption{PMAO and PASTA variants based on iteration count and guide tree decomposition strategy.}
    \begin{tabular}{c|l|r|l}
    \multicolumn{1}{l|}{Method} & Variant & \multicolumn{1}{l|}{Iter.} & Tree decomposition\\
    \hline
    \multirow{4}{*}{PMAO} & PMAO-3I-D  & 3     & Default (\textit{mincluster}) \\
\cline{2-4}          & 	PMAO-8I-D  & 8     & Default (\textit{mincluster}) \\
\cline{2-4}          & PMAO-3I-S  & 3     & Stochastic \\
\cline{2-4}          & PMAO-8I-S  & 8     & Stochastic\\
\hline \hline
    \multirow{4}{*}{PASTA} & PMAO-3I-D  & 3     & Default (\textit{mincluster}) \\
\cline{2-4}          & PASTA-8I-D  & 8     & Default (\textit{mincluster}) \\
\cline{2-4}          & PASTA-3I-S  & 3     & Stochastic \\
\cline{2-4}          & PASTA-8I-S  & 8     & Stochastic\\
    %\hline
    \end{tabular}\label{tab:variants}\end{table}


\begin{table*}[!htbp]
	\centering
	\small
%\begin{adjustwidth}{-0.6cm}{}
	\caption{Best FN rate achieved by the four variants of PMAO for each dataset in set A. On each dataset (i.e., row), the better (i.e., lower) FN rate values are marked with a darker shade.}
	\begin{tabular}{|l|r|r|r|r|}
		\hline
		\multirow{2}{*}{Dataset} & \multicolumn{4}{c|}{Best FN rates achieved by PMAO variants} \\
		\cline{2-5}          & PMAO-3I-D & PMAO-8I-D & PMAO-3I-S & PMAO-8I-S \\
		\hline
		BB11005 & \cellcolor[rgb]{ .988,  1,  .992}0.18 & \cellcolor[rgb]{ .384,  .745,  .478}0.09 & \cellcolor[rgb]{ .384,  .745,  .478}0.09 & \cellcolor[rgb]{ .384,  .745,  .478}0.09 \\
		\hline
		BB11018 & \cellcolor[rgb]{ .988,  1,  .992}0.27 & \cellcolor[rgb]{ .384,  .745,  .478}0.18 & \cellcolor[rgb]{ .384,  .745,  .478}0.18 & \cellcolor[rgb]{ .384,  .745,  .478}0.18 \\
		\hline
		BB11033 & \cellcolor[rgb]{ .988,  1,  .992}0.38 & \cellcolor[rgb]{ .988,  1,  .992}0.38 & \cellcolor[rgb]{ .988,  1,  .992}0.38 & \cellcolor[rgb]{ .988,  1,  .992}0.38 \\
		\hline
		BB11020 & \cellcolor[rgb]{ .988,  1,  .992}0.33 & \cellcolor[rgb]{ .988,  1,  .992}0.33 & \cellcolor[rgb]{ .988,  1,  .992}0.33 & \cellcolor[rgb]{ .988,  1,  .992}0.33 \\
		\hline
		BB12001 & \cellcolor[rgb]{ .988,  1,  .992}0.13 & \cellcolor[rgb]{ .988,  1,  .992}0.13 & \cellcolor[rgb]{ .988,  1,  .992}0.13 & \cellcolor[rgb]{ .988,  1,  .992}0.13 \\
		\hline
		BB12013 & \cellcolor[rgb]{ .384,  .745,  .478}0.00 & \cellcolor[rgb]{ .988,  1,  .992}0.20 & \cellcolor[rgb]{ .384,  .745,  .478}0.00 & \cellcolor[rgb]{ .988,  1,  .992}0.20 \\
		\hline
		BB12022 & \cellcolor[rgb]{ .988,  1,  .992}0.00 & \cellcolor[rgb]{ .988,  1,  .992}0.00 & \cellcolor[rgb]{ .988,  1,  .992}0.00 & \cellcolor[rgb]{ .988,  1,  .992}0.00 \\
		\hline
		BB12035 & \cellcolor[rgb]{ .988,  1,  .992}0.04 & \cellcolor[rgb]{ .384,  .745,  .478}0.00 & \cellcolor[rgb]{ .988,  1,  .992}0.04 & \cellcolor[rgb]{ .988,  1,  .992}0.04 \\
		\hline
		BB12044 & \cellcolor[rgb]{ .988,  1,  .992}0.38 & \cellcolor[rgb]{ .988,  1,  .992}0.38 & \cellcolor[rgb]{ .988,  1,  .992}0.38 & \cellcolor[rgb]{ .988,  1,  .992}0.38 \\
		\hline
		BB20001 & \cellcolor[rgb]{ .384,  .745,  .478}0.23 & \cellcolor[rgb]{ .988,  1,  .992}0.46 & \cellcolor[rgb]{ .384,  .745,  .478}0.23 & \cellcolor[rgb]{ .384,  .745,  .478}0.23 \\
		\hline
		BB20010 & \cellcolor[rgb]{ .988,  1,  .992}0.31 & \cellcolor[rgb]{ .988,  1,  .992}0.31 & \cellcolor[rgb]{ .384,  .745,  .478}0.08 & \cellcolor[rgb]{ .384,  .745,  .478}0.08 \\
		\hline
		BB20022 & \cellcolor[rgb]{ .988,  1,  .992}0.11 & \cellcolor[rgb]{ .988,  1,  .992}0.11 & \cellcolor[rgb]{ .988,  1,  .992}0.11 & \cellcolor[rgb]{ .384,  .745,  .478}0.09 \\
		\hline
		BB20033 & \cellcolor[rgb]{ .988,  1,  .992}0.36 & \cellcolor[rgb]{ .988,  1,  .992}0.36 & \cellcolor[rgb]{ .988,  1,  .992}0.36 & \cellcolor[rgb]{ .384,  .745,  .478}0.24 \\
		\hline
		BB20041 & \cellcolor[rgb]{ .988,  1,  .992}0.33 & \cellcolor[rgb]{ .584,  .827,  .647}0.29 & \cellcolor[rgb]{ .384,  .745,  .478}0.27 & \cellcolor[rgb]{ .988,  1,  .992}0.33 \\
		\hline
		BB30002 & \cellcolor[rgb]{ .988,  1,  .992}0.32 & \cellcolor[rgb]{ .988,  1,  .992}0.32 & \cellcolor[rgb]{ .384,  .745,  .478}0.14 & \cellcolor[rgb]{ .502,  .792,  .58}0.18 \\
		\hline
		BB30008 & \cellcolor[rgb]{ .533,  .808,  .604}0.24 & \cellcolor[rgb]{ .988,  1,  .992}0.33 & \cellcolor[rgb]{ .835,  .933,  .863}0.30 & \cellcolor[rgb]{ .384,  .745,  .478}0.21 \\
		\hline
		BB30015 & \cellcolor[rgb]{ .988,  1,  .992}0.17 & \cellcolor[rgb]{ .988,  1,  .992}0.17 & \cellcolor[rgb]{ .988,  1,  .992}0.17 & \cellcolor[rgb]{ .988,  1,  .992}0.17 \\
		\hline
		BB30022 & \cellcolor[rgb]{ .682,  .871,  .733}0.48 & \cellcolor[rgb]{ .682,  .871,  .733}0.48 & \cellcolor[rgb]{ .988,  1,  .992}0.49 & \cellcolor[rgb]{ .384,  .745,  .478}0.46 \\
		\hline
		BB40001 & \cellcolor[rgb]{ .988,  1,  .992}0.48 & \cellcolor[rgb]{ .784,  .914,  .82}0.44 & \cellcolor[rgb]{ .988,  1,  .992}0.48 & \cellcolor[rgb]{ .384,  .745,  .478}0.36 \\
		\hline
		BB40013 & \cellcolor[rgb]{ .988,  1,  .992}0.31 & \cellcolor[rgb]{ .988,  1,  .992}0.31 & \cellcolor[rgb]{ .988,  1,  .992}0.31 & \cellcolor[rgb]{ .384,  .745,  .478}0.25 \\
		\hline
		BB40025 & \cellcolor[rgb]{ .988,  1,  .992}0.00 & \cellcolor[rgb]{ .988,  1,  .992}0.00 & \cellcolor[rgb]{ .988,  1,  .992}0.00 & \cellcolor[rgb]{ .988,  1,  .992}0.00 \\
		\hline
		BB40038 & \cellcolor[rgb]{ .988,  1,  .992}0.10 & \cellcolor[rgb]{ .988,  1,  .992}0.10 & \cellcolor[rgb]{ .988,  1,  .992}0.10 & \cellcolor[rgb]{ .988,  1,  .992}0.10 \\
		\hline
		BB40048 & \cellcolor[rgb]{ .988,  1,  .992}0.29 & \cellcolor[rgb]{ .988,  1,  .992}0.29 & \cellcolor[rgb]{ .384,  .745,  .478}0.21 & \cellcolor[rgb]{ .988,  1,  .992}0.29 \\
		\hline
		BB50001 & \cellcolor[rgb]{ .988,  1,  .992}0.29 & \cellcolor[rgb]{ .988,  1,  .992}0.29 & \cellcolor[rgb]{ .988,  1,  .992}0.29 & \cellcolor[rgb]{ .988,  1,  .992}0.29 \\
		\hline
		BB50005 & \cellcolor[rgb]{ .686,  .871,  .733}0.25 & \cellcolor[rgb]{ .988,  1,  .992}0.38 & \cellcolor[rgb]{ .384,  .745,  .478}0.13 & \cellcolor[rgb]{ .686,  .871,  .733}0.25 \\
		\hline
		BB50010 & \cellcolor[rgb]{ .988,  1,  .992}0.07 & \cellcolor[rgb]{ .384,  .745,  .478}0.00 & \cellcolor[rgb]{ .384,  .745,  .478}0.00 & \cellcolor[rgb]{ .384,  .745,  .478}0.00 \\
		\hline
		BB50016 & \cellcolor[rgb]{ .682,  .871,  .733}0.13 & \cellcolor[rgb]{ .988,  1,  .992}0.20 & \cellcolor[rgb]{ .384,  .745,  .478}0.07 & \cellcolor[rgb]{ .384,  .745,  .478}0.07 \\
		\hline
	\end{tabular}\label{tab:pmao-variants-a}
%\end{adjustwidth}
\end{table*}

\begin{table*}[!htbp]
	\small
\caption{Best FN rate achieved by the four variants of PMAO for each dataset in set B. On each dataset (i.e., row), the better (i.e., lower) FN rate values are marked with a darker shade.}
	\begin{tabular}{|l|r|r|r|r|}
		\hline
		\multirow{2}{*}{Dataset} & \multicolumn{4}{c|}{Best FN rates achieved by PMAO variants} \\
		\cline{2-5}          & \multicolumn{1}{l|}{PMAO-3I-D} & \multicolumn{1}{l|}{PMAO-8I-D} & \multicolumn{1}{l|}{PMAO-3I-S} & \multicolumn{1}{l|}{PMAO-8I-S} \\
		\hline
		BB11007 & \cellcolor[rgb]{ .384,  .745,  .478}0.33 & \cellcolor[rgb]{ .384,  .745,  .478}0.33 & \cellcolor[rgb]{ .988,  1,  .992}0.50 & \cellcolor[rgb]{ .384,  .745,  .478}0.33 \\
		\hline
		BB11034 & \cellcolor[rgb]{ .988,  1,  .992}0.20 & \cellcolor[rgb]{ .384,  .745,  .478}0.00 & \cellcolor[rgb]{ .988,  1,  .992}0.20 & \cellcolor[rgb]{ .988,  1,  .992}0.20 \\
		\hline
		BB11038 & \cellcolor[rgb]{ .384,  .745,  .478}0.00 & \cellcolor[rgb]{ .384,  .745,  .478}0.00 & \cellcolor[rgb]{ .988,  1,  .992}0.40 & \cellcolor[rgb]{ .384,  .745,  .478}0.00 \\
		\hline
		BB11019 & \cellcolor[rgb]{ .988,  1,  .992}0.14 & \cellcolor[rgb]{ .988,  1,  .992}0.14 & \cellcolor[rgb]{ .988,  1,  .992}0.14 & \cellcolor[rgb]{ .988,  1,  .992}0.14 \\
		\hline
		BB12005 & \cellcolor[rgb]{ .988,  1,  .992}0.33 & \cellcolor[rgb]{ .988,  1,  .992}0.33 & \cellcolor[rgb]{ .988,  1,  .992}0.33 & \cellcolor[rgb]{ .988,  1,  .992}0.33 \\
		\hline
		BB12029 & \cellcolor[rgb]{ .988,  1,  .992}0.44 & \cellcolor[rgb]{ .988,  1,  .992}0.44 & \cellcolor[rgb]{ .384,  .745,  .478}0.33 & \cellcolor[rgb]{ .384,  .745,  .478}0.33 \\
		\hline
		BB12026 & \cellcolor[rgb]{ .384,  .745,  .478}0.27 & \cellcolor[rgb]{ .384,  .745,  .478}0.27 & \cellcolor[rgb]{ .384,  .745,  .478}0.27 & \cellcolor[rgb]{ .988,  1,  .992}0.33 \\
		\hline
		BB12037 & \cellcolor[rgb]{ .988,  1,  .992}0.10 & \cellcolor[rgb]{ .988,  1,  .992}0.10 & \cellcolor[rgb]{ .988,  1,  .992}0.10 & \cellcolor[rgb]{ .988,  1,  .992}0.10 \\
		\hline
		BB20002 & \cellcolor[rgb]{ .686,  .871,  .733}0.53 & \cellcolor[rgb]{ .384,  .745,  .478}0.47 & \cellcolor[rgb]{ .988,  1,  .992}0.59 & \cellcolor[rgb]{ .686,  .871,  .733}0.53 \\
		\hline
		BB20012 & \cellcolor[rgb]{ .988,  1,  .992}0.29 & \cellcolor[rgb]{ .745,  .894,  .784}0.21 & \cellcolor[rgb]{ .745,  .894,  .784}0.21 & \cellcolor[rgb]{ .384,  .745,  .478}0.08 \\
		\hline
		BB20030 & \cellcolor[rgb]{ .988,  1,  .992}0.64 & \cellcolor[rgb]{ .384,  .745,  .478}0.50 & \cellcolor[rgb]{ .988,  1,  .992}0.64 & \cellcolor[rgb]{ .988,  1,  .992}0.64 \\
		\hline
		BB20037 & \cellcolor[rgb]{ .988,  1,  .992}0.13 & \cellcolor[rgb]{ .988,  1,  .992}0.13 & \cellcolor[rgb]{ .988,  1,  .992}0.13 & \cellcolor[rgb]{ .988,  1,  .992}0.13 \\
		\hline
		BB30003 & \cellcolor[rgb]{ .988,  1,  .992}0.27 & \cellcolor[rgb]{ .384,  .745,  .478}0.25 & \cellcolor[rgb]{ .682,  .871,  .733}0.26 & \cellcolor[rgb]{ .682,  .871,  .733}0.26 \\
		\hline
		BB30021 & \cellcolor[rgb]{ .729,  .89,  .769}0.30 & \cellcolor[rgb]{ .816,  .925,  .843}0.31 & \cellcolor[rgb]{ .384,  .745,  .478}0.27 & \cellcolor[rgb]{ .988,  1,  .992}0.32 \\
		\hline
		BB30026 & \cellcolor[rgb]{ .384,  .745,  .478}0.15 & \cellcolor[rgb]{ .384,  .745,  .478}0.15 & \cellcolor[rgb]{ .988,  1,  .992}0.17 & \cellcolor[rgb]{ .384,  .745,  .478}0.15 \\
		\hline
		BB30011 & \cellcolor[rgb]{ .988,  1,  .992}0.32 & \cellcolor[rgb]{ .384,  .745,  .478}0.31 & \cellcolor[rgb]{ .384,  .745,  .478}0.31 & \cellcolor[rgb]{ .384,  .745,  .478}0.31 \\
		\hline
		BB40009 & \cellcolor[rgb]{ .988,  1,  .992}0.13 & \cellcolor[rgb]{ .988,  1,  .992}0.13 & \cellcolor[rgb]{ .988,  1,  .992}0.13 & \cellcolor[rgb]{ .988,  1,  .992}0.13 \\
		\hline
		BB40019 & \cellcolor[rgb]{ .988,  1,  .992}0.29 & \cellcolor[rgb]{ .988,  1,  .992}0.29 & \cellcolor[rgb]{ .384,  .745,  .478}0.14 & \cellcolor[rgb]{ .988,  1,  .992}0.29 \\
		\hline
		BB40033 & \cellcolor[rgb]{ .988,  1,  .992}0.06 & \cellcolor[rgb]{ .988,  1,  .992}0.06 & \cellcolor[rgb]{ .988,  1,  .992}0.06 & \cellcolor[rgb]{ .988,  1,  .992}0.06 \\
		\hline
		BB40006 & \cellcolor[rgb]{ .988,  1,  .992}0.00 & \cellcolor[rgb]{ .988,  1,  .992}0.00 & \cellcolor[rgb]{ .988,  1,  .992}0.00 & \cellcolor[rgb]{ .988,  1,  .992}0.00 \\
		\hline
		BB50002 & \cellcolor[rgb]{ .384,  .745,  .478}0.40 & \cellcolor[rgb]{ .988,  1,  .992}0.50 & \cellcolor[rgb]{ .988,  1,  .992}0.50 & \cellcolor[rgb]{ .384,  .745,  .478}0.40 \\
		\hline
		BB50009 & \cellcolor[rgb]{ .988,  1,  .992}0.08 & \cellcolor[rgb]{ .988,  1,  .992}0.08 & \cellcolor[rgb]{ .988,  1,  .992}0.08 & \cellcolor[rgb]{ .988,  1,  .992}0.08 \\
		\hline
		BB50014 & \cellcolor[rgb]{ .988,  1,  .992}0.41 & \cellcolor[rgb]{ .988,  1,  .992}0.41 & \cellcolor[rgb]{ .988,  1,  .992}0.41 & \cellcolor[rgb]{ .384,  .745,  .478}0.26 \\
		\hline
		BB50006 & \cellcolor[rgb]{ .988,  1,  .992}0.16 & \cellcolor[rgb]{ .988,  1,  .992}0.16 & \cellcolor[rgb]{ .988,  1,  .992}0.16 & \cellcolor[rgb]{ .384,  .745,  .478}0.12 \\
		\hline
	\end{tabular}\label{tab:pmao-variants-b}
\end{table*}

\begin{table*}[!htbp]
	\small
	\caption{\underline{Friedman Aligned Ranks test (Column 2):} Friedman Aligned ranks (lower is better) of the four variants of PMAO based on Table~\ref{tab:pmao-variants-a},  \ref{tab:pmao-variants-b}. We also show the computed statistics and corresponding $ p $-value.
		\underline{Holm's post-hoc procedure (Columns 3 - 6):} Comparison among the PMAO variants using the Holm's post-hoc procedures. Each entry shows the adjusted $p$-value which indicates the significance of the difference in performance between two methods. The significant differences are marked with green color.}
	\begin{tabular}{|l|r||c|c|c|c|}
		\hline
		\multicolumn{1}{|c|}{1} & \multicolumn{1}{c||}{2} & \multicolumn{1}{c|}{3} & \multicolumn{1}{c|}{4} & \multicolumn{1}{c|}{5} & 6 \\
		\hline
		\multirow{2}{*}{\makecell{PMAO\\ variants}} & \multirow{2}{*}{\makecell{Friedman\\Aligned rank*}} & \multicolumn{4}{c|}{Holm's adjusted $p$-value} \\
		\cline{3-6}          &       & 8I-S & 3I-S & 8I-D & 3I-D \\
		\hline
		8I-S & 82.3137 & \multicolumn{1}{c|}{-} & \multicolumn{1}{r|}{0.4032} & \multicolumn{1}{r|}{0.1431} & \multicolumn{1}{r|}{\cellcolor[rgb]{ .384,  .745,  .478}0.0077} \\
		\hline
		3I-S & 99.8137 & \multicolumn{1}{r|}{0.4032} & \multicolumn{1}{c|}{-} & \multicolumn{1}{r|}{0.6038} & \multicolumn{1}{r|}{0.3387} \\
		\hline
		8I-D & 107.9020 & \multicolumn{1}{r|}{0.1431} & \multicolumn{1}{r|}{0.6038} & \multicolumn{1}{c|}{-} & \multicolumn{1}{r|}{0.6038} \\
		\hline
		3I-D & 119.9706 & \multicolumn{1}{r|}{\cellcolor[rgb]{ .384,  .745,  .478}0.0077} & \multicolumn{1}{r|}{0.3387} & \multicolumn{1}{r|}{0.6038} & - \\
		\hline \hline
		*Statistic & 9.0393 & \multicolumn{4}{c|}{\multirow{2}{*}{N/A}} \\
		\cline{1-2}    *$p$-value & 0.0288 & \multicolumn{4}{c|}{} \\
		\hline
	\end{tabular}\label{tab:test-pmao-variants}
\end{table*}

\subsection{Comparison between PASTA and PMAO}
We compare the solutions generated by PMAO (8I-S variant) with the output of PASTA (8I-D variant) in terms of FN rate. In particular, we report the PASTA FN rate alongside the best FN rate achieved by PMAO in Table~\ref{tab:pmao-pasta-a} and \ref{tab:pmao-pasta-b} where on each row the better values are marked with a darker shade. Furthermore, to give an overall picture of the tree-space generated by the PMAO framework, we include the average FN rate and count of those PMAO solutions equivalent to or better than PASTA. Across the rightmost column (i.e., count), the better (i.e., greater) values are marked with a darker shade. Except for BB12035, in every case, PMAO has at least one solution better than or equivalent to PASTA. Note that there are several solutions better than PASTA in the tree-space comprising 30 solutions and in 39 cases out of 51, more than 10 solutions are better than PASTA. We find that the best FN rates of PMAO are clearly ahead of PASTA's FN rate. In several cases the improvement is more than 50\% (e.g., BB11020, Bb20001, BB20010, BB40028, BB50016, BB11038, BB12037, BB20012, BB40033, BB50009, etc.). These results demonstrate the advantages of PMAO over PASTA.

We present similar results considering all BAliBASE 3.0 datasets in Supplementary Table~\ref{tab:pmao-pasta-all} where in every case, PMAO has at least one solution better than or equivalent to PASTA. Furthermore, more than 10 solutions (out of 30) are better than PASTA in 174 cases out of 218, and in 47 cases, the improvement of FN rate is more than 50\%.
\begin{table}[!htbp]
\small
\caption{Comparison of the 30 solutions generated by PMAO with respect to PASTA in terms of FN rate on set A datasets. For PMAO, we show the best FN rate along with the average FN rateand count of its solutions better or equivalent to PASTA. On each dataset (i.e., row), the better (i.e., lower) FN rate values are marked with a darker shade. Across the rightmost column (i.e., count), the better (i.e., greater) values are marked with a darker shade. }
	\begin{tabular}{|l|r|r|r||r|}
		\hline
		\multirow{2}{*}{Dataset} & \multirow{2}{*}{\makecell{PASTA\\FN rate}} & \multicolumn{3}{c|}{\makecell{PMAO solutions better \\or equivalent to PASTA}} \\
		\cline{3-5}          &       & \multicolumn{1}{l|}{Best FN} & \multicolumn{1}{l|}{Avg FN} & \multicolumn{1}{l|}{Count} \\
		\hline
		BB11005 & \cellcolor[rgb]{ .988,  .988,  1}0.55 & \cellcolor[rgb]{ .388,  .745,  .482}0.09 & \cellcolor[rgb]{ .753,  .894,  .8}0.37 & \cellcolor[rgb]{ .976,  .451,  .459}28 \\
		\hline
		BB11018 & \cellcolor[rgb]{ .988,  .988,  1}0.27 & \cellcolor[rgb]{ .388,  .745,  .482}0.18 & \cellcolor[rgb]{ .784,  .906,  .824}0.24 & \cellcolor[rgb]{ .988,  .875,  .886}6 \\
		\hline
		BB11033 & \cellcolor[rgb]{ .988,  .988,  1}0.38 & \cellcolor[rgb]{ .988,  .988,  1}0.38 & \cellcolor[rgb]{ .988,  .988,  1}0.38 & \cellcolor[rgb]{ .988,  .894,  .906}5 \\
		\hline
		BB11020 & \cellcolor[rgb]{ .988,  .988,  1}0.83 & \cellcolor[rgb]{ .388,  .745,  .482}0.33 & \cellcolor[rgb]{ .733,  .882,  .78}0.62 & \cellcolor[rgb]{ .973,  .412,  .42}30 \\
		\hline
		BB12001 & \cellcolor[rgb]{ .988,  .988,  1}0.25 & \cellcolor[rgb]{ .388,  .745,  .482}0.13 & \cellcolor[rgb]{ .906,  .953,  .929}0.23 & \cellcolor[rgb]{ .98,  .702,  .71}15 \\
		\hline
		BB12013 & \cellcolor[rgb]{ .988,  .988,  1}0.20 & \cellcolor[rgb]{ .988,  .988,  1}0.20 & \cellcolor[rgb]{ .988,  .988,  1}0.20 & \cellcolor[rgb]{ .973,  .412,  .42}30 \\
		\hline
		BB12022 & \cellcolor[rgb]{ .988,  .988,  1}0.00 & \cellcolor[rgb]{ .988,  .988,  1}0.00 & \cellcolor[rgb]{ .988,  .988,  1}0.00 & \cellcolor[rgb]{ .976,  .51,  .518}25 \\
		\hline
		BB12035 & \cellcolor[rgb]{ .388,  .745,  .482}0.00  &   0.04    &   -    & \cellcolor[rgb]{ .988,  .988,  1}0 \\
		\hline
		BB12044 & \cellcolor[rgb]{ .988,  .988,  1}0.50 & \cellcolor[rgb]{ .388,  .745,  .482}0.38 & \cellcolor[rgb]{ .706,  .875,  .757}0.44 & \cellcolor[rgb]{ .973,  .412,  .42}30 \\
		\hline
		BB20001 & \cellcolor[rgb]{ .988,  .988,  1}0.54 & \cellcolor[rgb]{ .388,  .745,  .482}0.23 & \cellcolor[rgb]{ .843,  .929,  .875}0.46 & \cellcolor[rgb]{ .976,  .51,  .518}25 \\
		\hline
		BB20010 & \cellcolor[rgb]{ .988,  .988,  1}0.35 & \cellcolor[rgb]{ .388,  .745,  .482}0.08 & \cellcolor[rgb]{ .871,  .941,  .898}0.29 & \cellcolor[rgb]{ .976,  .549,  .557}23 \\
		\hline
		BB20022 & \cellcolor[rgb]{ .988,  .988,  1}0.11 & \cellcolor[rgb]{ .388,  .745,  .482}0.09 & \cellcolor[rgb]{ .925,  .961,  .945}0.11 & \cellcolor[rgb]{ .984,  .796,  .808}10 \\
		\hline
		BB20033 & \cellcolor[rgb]{ .988,  .988,  1}0.36 & \cellcolor[rgb]{ .388,  .745,  .482}0.24 & \cellcolor[rgb]{ .784,  .906,  .824}0.32 & \cellcolor[rgb]{ .984,  .816,  .827}9 \\
		\hline
		BB20041 & \cellcolor[rgb]{ .988,  .988,  1}0.38 & \cellcolor[rgb]{ .388,  .745,  .482}0.33 & \cellcolor[rgb]{ .8,  .91,  .835}0.36 & \cellcolor[rgb]{ .984,  .835,  .847}8 \\
		\hline
		BB30002 & \cellcolor[rgb]{ .988,  .988,  1}0.32 & \cellcolor[rgb]{ .388,  .745,  .482}0.18 & \cellcolor[rgb]{ .847,  .929,  .878}0.29 & \cellcolor[rgb]{ .98,  .702,  .71}15 \\
		\hline
		BB30008 & \cellcolor[rgb]{ .988,  .988,  1}0.33 & \cellcolor[rgb]{ .388,  .745,  .482}0.21 & \cellcolor[rgb]{ .878,  .941,  .906}0.31 & \cellcolor[rgb]{ .984,  .722,  .729}14 \\
		\hline
		BB30015 & \cellcolor[rgb]{ .988,  .988,  1}0.17 & \cellcolor[rgb]{ .988,  .988,  1}0.17 & \cellcolor[rgb]{ .988,  .988,  1}0.17 & \cellcolor[rgb]{ .984,  .761,  .769}12 \\
		\hline
		BB30022 & \cellcolor[rgb]{ .988,  .988,  1}0.51 & \cellcolor[rgb]{ .388,  .745,  .482}0.46 & \cellcolor[rgb]{ .902,  .953,  .925}0.50 & \cellcolor[rgb]{ .98,  .663,  .675}17 \\
		\hline
		BB40001 & \cellcolor[rgb]{ .988,  .988,  1}0.48 & \cellcolor[rgb]{ .388,  .745,  .482}0.36 & \cellcolor[rgb]{ .745,  .89,  .792}0.43 & \cellcolor[rgb]{ .984,  .796,  .808}10 \\
		\hline
		BB40013 & \cellcolor[rgb]{ .988,  .988,  1}0.38 & \cellcolor[rgb]{ .388,  .745,  .482}0.25 & \cellcolor[rgb]{ .753,  .89,  .796}0.33 & \cellcolor[rgb]{ .98,  .682,  .694}16 \\
		\hline
		BB40025 & \cellcolor[rgb]{ .988,  .988,  1}0.00 & \cellcolor[rgb]{ .988,  .988,  1}0.00 & \cellcolor[rgb]{ .988,  .988,  1}0.00 & \cellcolor[rgb]{ .98,  .569,  .576}22 \\
		\hline
		BB40038 & \cellcolor[rgb]{ .988,  .988,  1}0.25 & \cellcolor[rgb]{ .388,  .745,  .482}0.10 & \cellcolor[rgb]{ .788,  .906,  .827}0.20 & \cellcolor[rgb]{ .98,  .624,  .635}19 \\
		\hline
		BB40048 & \cellcolor[rgb]{ .988,  .988,  1}0.43 & \cellcolor[rgb]{ .388,  .745,  .482}0.29 & \cellcolor[rgb]{ .447,  .769,  .533}0.30 & \cellcolor[rgb]{ .973,  .412,  .42}30 \\
		\hline
		BB50001 & \cellcolor[rgb]{ .988,  .988,  1}0.29 & \cellcolor[rgb]{ .988,  .988,  1}0.29 & \cellcolor[rgb]{ .988,  .988,  1}0.29 & \cellcolor[rgb]{ .98,  .663,  .675}17 \\
		\hline
		BB50005 & \cellcolor[rgb]{ .988,  .988,  1}0.38 & \cellcolor[rgb]{ .388,  .745,  .482}0.25 & \cellcolor[rgb]{ .827,  .922,  .859}0.34 & \cellcolor[rgb]{ .973,  .412,  .42}30 \\
		\hline
		BB50010 & \cellcolor[rgb]{ .988,  .988,  1}0.00 & \cellcolor[rgb]{ .988,  .988,  1}0.00 & \cellcolor[rgb]{ .988,  .988,  1}0.00 & \cellcolor[rgb]{ .988,  .855,  .867}7 \\
		\hline
		BB50016 & \cellcolor[rgb]{ .988,  .988,  1}0.47 & \cellcolor[rgb]{ .388,  .745,  .482}0.07 & \cellcolor[rgb]{ .584,  .824,  .651}0.20 & \cellcolor[rgb]{ .976,  .451,  .459}28 \\
		\hline
	\end{tabular}\label{tab:pmao-pasta-a}\end{table}

\begin{table}[!htbp]
	\small
	\caption{Comparison of the 30 solutions generated by PMAO with respect to PASTA in terms of FN rate on set B datasets. For PMAO, we show the best FN rate along with the average FN rate and count of its solutions better or equivalent to PASTA. On each dataset (i.e., row), the better (i.e., lower) FN rate values are marked with a darker shade. Across the rightmost column (i.e., count), the better (i.e., greater) values are marked with a darker shade.}
	\begin{tabular}{|l|r|r|r||r|}
		\hline
		\multirow{2}{*}{Dataset} & \multirow{2}{*}{\makecell{PASTA\\FN rate}} & \multicolumn{3}{c|}{\makecell{PMAO solutions better \\or equivalent to PASTA}} \\
		\cline{3-5}          &       & \multicolumn{1}{l|}{Best FN} & \multicolumn{1}{l|}{Avg FN} & \multicolumn{1}{l|}{Count} \\
		\hline
		BB11007 & \cellcolor[rgb]{ .988,  1,  .992}0.50 & \cellcolor[rgb]{ .384,  .745,  .478}0.33 & \cellcolor[rgb]{ .906,  .965,  .922}0.48 & \cellcolor[rgb]{ .984,  .714,  .722}15 \\
		\hline
		BB11034 & \cellcolor[rgb]{ .988,  1,  .992}0.40 & \cellcolor[rgb]{ .384,  .745,  .478}0.20 & \cellcolor[rgb]{ .753,  .898,  .792}0.32 & \cellcolor[rgb]{ .98,  .651,  .663}18 \\
		\hline
		BB11038 & \cellcolor[rgb]{ .988,  1,  .992}0.40 & \cellcolor[rgb]{ .384,  .745,  .478}0.00 & \cellcolor[rgb]{ .937,  .976,  .949}0.37 & \cellcolor[rgb]{ .984,  .773,  .78}12 \\
		\hline
		BB11019 & \cellcolor[rgb]{ .988,  1,  .992}0.29 & \cellcolor[rgb]{ .384,  .745,  .478}0.14 & \cellcolor[rgb]{ .918,  .969,  .933}0.27 & \cellcolor[rgb]{ .976,  .475,  .482}27 \\
		\hline
		BB12005 & \cellcolor[rgb]{ .988,  1,  .992}0.33 & \cellcolor[rgb]{ .988,  1,  .992}0.33 & \cellcolor[rgb]{ .988,  1,  .992}0.33 & \cellcolor[rgb]{ .976,  .455,  .463}28 \\
		\hline
		BB12029 & \cellcolor[rgb]{ .988,  1,  .992}0.44 & \cellcolor[rgb]{ .384,  .745,  .478}0.33 & \cellcolor[rgb]{ .863,  .945,  .882}0.42 & \cellcolor[rgb]{ .976,  .435,  .443}29 \\
		\hline
		BB12026 & \cellcolor[rgb]{ .988,  1,  .992}0.53 & \cellcolor[rgb]{ .384,  .745,  .478}0.33 & \cellcolor[rgb]{ .816,  .925,  .847}0.48 & \cellcolor[rgb]{ .984,  .753,  .761}13 \\
		\hline
		BB12037 & \cellcolor[rgb]{ .988,  1,  .992}0.40 & \cellcolor[rgb]{ .384,  .745,  .478}0.10 & \cellcolor[rgb]{ .859,  .945,  .882}0.34 & \cellcolor[rgb]{ .988,  .851,  .863}8 \\
		\hline
		BB20002 & \cellcolor[rgb]{ .988,  1,  .992}0.65 & \cellcolor[rgb]{ .384,  .745,  .478}0.53 & \cellcolor[rgb]{ .835,  .933,  .863}0.62 & \cellcolor[rgb]{ .984,  .812,  .824}10 \\
		\hline
		BB20012 & \cellcolor[rgb]{ .988,  1,  .992}0.29 & \cellcolor[rgb]{ .384,  .745,  .478}0.08 & \cellcolor[rgb]{ .851,  .941,  .875}0.25 & \cellcolor[rgb]{ .976,  .494,  .502}26 \\
		\hline
		BB20030 & \cellcolor[rgb]{ .988,  1,  .992}0.64 & \cellcolor[rgb]{ .988,  1,  .992}0.64 & \cellcolor[rgb]{ .988,  1,  .992}0.64 & \cellcolor[rgb]{ .988,  .871,  .882}7 \\
		\hline
		BB20037 & \cellcolor[rgb]{ .988,  1,  .992}0.13 & \cellcolor[rgb]{ .988,  1,  .992}0.13 & \cellcolor[rgb]{ .988,  1,  .992}0.13 & \cellcolor[rgb]{ .988,  .988,  1}1 \\
		\hline
		BB30003 & \cellcolor[rgb]{ .988,  1,  .992}0.30 & \cellcolor[rgb]{ .384,  .745,  .478}0.26 & \cellcolor[rgb]{ .773,  .906,  .808}0.29 & \cellcolor[rgb]{ .98,  .573,  .58}22 \\
		\hline
		BB30021 & \cellcolor[rgb]{ .988,  1,  .992}0.36 & \cellcolor[rgb]{ .384,  .745,  .478}0.32 & \cellcolor[rgb]{ .663,  .863,  .714}0.34 & \cellcolor[rgb]{ .98,  .69,  .702}16 \\
		\hline
		BB30026 & \cellcolor[rgb]{ .988,  1,  .992}0.19 & \cellcolor[rgb]{ .384,  .745,  .478}0.15 & \cellcolor[rgb]{ .741,  .894,  .78}0.18 & \cellcolor[rgb]{ .984,  .831,  .843}9 \\
		\hline
		BB30011 & \cellcolor[rgb]{ .988,  1,  .992}0.35 & \cellcolor[rgb]{ .384,  .745,  .478}0.31 & \cellcolor[rgb]{ .894,  .957,  .91}0.35 & \cellcolor[rgb]{ .973,  .412,  .42}30 \\
		\hline
		BB40009 & \cellcolor[rgb]{ .988,  1,  .992}0.19 & \cellcolor[rgb]{ .384,  .745,  .478}0.13 & \cellcolor[rgb]{ .945,  .98,  .957}0.18 & \cellcolor[rgb]{ .984,  .714,  .722}15 \\
		\hline
		BB40019 & \cellcolor[rgb]{ .988,  1,  .992}0.43 & \cellcolor[rgb]{ .384,  .745,  .478}0.29 & \cellcolor[rgb]{ .725,  .886,  .769}0.37 & \cellcolor[rgb]{ .973,  .412,  .42}30 \\
		\hline
		BB40033 & \cellcolor[rgb]{ .988,  1,  .992}0.13 & \cellcolor[rgb]{ .384,  .745,  .478}0.06 & \cellcolor[rgb]{ .706,  .878,  .753}0.10 & \cellcolor[rgb]{ .976,  .455,  .463}28 \\
		\hline
		BB40006 & \cellcolor[rgb]{ .988,  1,  .992}0.00 & \cellcolor[rgb]{ .988,  1,  .992}0.00 & \cellcolor[rgb]{ .988,  1,  .992}0.00 & \cellcolor[rgb]{ .988,  .871,  .882}7 \\
		\hline
		BB50002 & \cellcolor[rgb]{ .988,  1,  .992}0.50 & \cellcolor[rgb]{ .384,  .745,  .478}0.40 & \cellcolor[rgb]{ .941,  .98,  .953}0.49 & \cellcolor[rgb]{ .984,  .733,  .741}14 \\
		\hline
		BB50009 & \cellcolor[rgb]{ .988,  1,  .992}0.24 & \cellcolor[rgb]{ .384,  .745,  .478}0.08 & \cellcolor[rgb]{ .788,  .914,  .82}0.19 & \cellcolor[rgb]{ .98,  .631,  .643}19 \\
		\hline
		BB50014 & \cellcolor[rgb]{ .988,  1,  .992}0.41 & \cellcolor[rgb]{ .384,  .745,  .478}0.26 & \cellcolor[rgb]{ .835,  .933,  .863}0.37 & \cellcolor[rgb]{ .984,  .812,  .824}10 \\
		\hline
		BB50006 & \cellcolor[rgb]{ .988,  1,  .992}0.21 & \cellcolor[rgb]{ .384,  .745,  .478}0.12 & \cellcolor[rgb]{ .725,  .886,  .769}0.17 & \cellcolor[rgb]{ .988,  .89,  .902}6 \\
		\hline
	\end{tabular}\label{tab:pmao-pasta-b}\end{table}

\subsubsection{Domain-specific measures and multiple objectives}
To illustrate a particular advantage of PMAO over PASTA, we plot the (ML score, FN rate) values of PASTA output and 30 solutions in the tree-space generated by PMAO in Figure~\ref{fig:ml-fn} for 10 arbitrary datasets. Such plot for all datasets of set A and set B are available in Supplementary Figures~\ref{fig:ml-fn-a}, \ref{fig:ml-fn-b}. We find several cases where a PMAO solution has a better ML score than PASTA but with a worse FN rate.  On the other hand, the best FN rate might correspond to an ML score much lower than PASTA. Similar observations can be shown for other objectives as well. These results illustrate the phenomena that, depending solely on a single optimization criterion (e.g., ML score by PASTA) may severely affect the accuracy. PMAO reduces such risk by employing many objectives.

\begin{figure*}[pos=!htbp, align=\centering, width=17cm]
	\centering
	\begin{adjustwidth}{-1cm}{}
		\subfloat[BB11018]{\includegraphics[width=0.22\textwidth]{BB11018-fn-ml}}
		\subfloat[BB12001]{\includegraphics[width=0.22\textwidth]{BB12001-fn-ml}}
		\subfloat[BB20010]{\includegraphics[width=0.22\textwidth]{BB20010-fn-ml}}
		\subfloat[BB20041]{\includegraphics[width=0.22\textwidth]{BB20041-fn-ml}}
		\subfloat[BB30002]{\includegraphics[width=0.22\textwidth]{BB30002-fn-ml}}\\
		\subfloat[BB30008]{\includegraphics[width=0.22\textwidth]{BB30008-fn-ml}}
		\subfloat[BB40001]{\includegraphics[width=0.22\textwidth]{BB40001-fn-ml}}
		\subfloat[BB40048]{\includegraphics[width=0.22\textwidth]{BB40048-fn-ml}}
		\subfloat[BB50005]{\includegraphics[width=0.22\textwidth]{BB50005-fn-ml}}
		\subfloat[BB50016]{\includegraphics[width=0.22\textwidth]{BB50016-fn-ml}}
		
	\caption{Visualization of PASTA output and the 30 solutions generated by PMAO on 10 arbitrary datasets. The x-axis and y-axis represent ML score and FN rate respectively.}
	\label{fig:ml-fn}
	\end{adjustwidth}
\end{figure*}

\subsubsection{Contribution of MO strategy}
Now we examine the flexibility of decomposition-based MO strategy within the PMAO framework to address diverse characteristics varying across datasets. The weight vectors that resulted in better than or equivalent solutions than that of PASTA in four arbitrary datasets are illustrated in Figures~\ref{fig:good-weight}. Here each bar represents a weight vector along with its achieved FN rate.  Note that the weight vectors yielding the best FN rates vary across the datasets suggesting that the underlying MO mechanism is somewhat self-adapting itself according to the inherent characteristics of the dataset. This is all the more evident when the number of weight vectors is increased (please refer to Supplementary Table~\ref{tab:pmao-100}). Additionally, observe that nearly all weight vectors consist of two or more non-zero values indicating the contribution of multiple objectives in the overall optimization process. These justify our motivation for employing the MO approach incorporating multiple weight vectors, which ensures the robustness of PMAO framework to varying traits of the datasets. Similar plots for all datasets of set A and set B are available in Supplementary Figures~\ref{fig:good-weight-a}, \ref{fig:good-weight-b}.

\begin{figure}[!htbp]\begin{adjustwidth}{-0.6cm}{}
		\centering
		\subfloat[BB12001]{\includegraphics[width=0.25\textwidth]{BB12001-good-weight}}
		\subfloat[BB20041]{\includegraphics[width=0.25\textwidth]{BB20041-good-weight}}\\
		\subfloat[BB30008]{\includegraphics[width=0.25\textwidth]{BB30008-good-weight}}
		\subfloat[BB40001]{\includegraphics[width=0.25\textwidth]{BB40001-good-weight}}
\end{adjustwidth}
	\caption{Visualization of weight vectors which lead PMAO to generate better or equivalent solutions to PASTA on four arbitrary datasets. The y-axis portrays the weight values and the x-axis marks the achieved FN rate. The weight vectors are sorted in ascending order based on the achieved FN rate. }
	\label{fig:good-weight}
\end{figure}




\subsection{Machine learning based detection of few wight vectors}
One awkward issue is that the MO optimization approaches usually provide a bunch of good quality solutions that are considered equivalent in the context of conflicting objectives under consideration. In MO terms, these are non-dominated solutions. However, from a practical point of view, it is often necessary to choose one of these as the final solution. While picking randomly could be an approach, but as experimental evidence suggests, if is unlikely that all solutions output by our PMAO framework for a particular input would be of the highest quality. So it is interesting to develop an approach that would help us choose the best one or at least a few top solutions from the output of PMAO. Such an approach would be of independent interest for other MO optimization scenarios as well.

With the above backdrop, now we address whether it is possible to detect a few input weight vectors for PMAO, based on the features of the unaligned input sequences, such that the resultant smaller tree-space contains at least one high-quality tree. It can significantly reduce PAMO's computational cost as well as allow the domain expert to select the final solution with a manageable effort. We design a supervised regression-based pipeline to detect five promising weight vectors comprising the following steps to pursue this goal.
\begin{itemize}
	\item Step 1: Prepare training data to learn a regression model of the form, $FN\_rate = f(X)$, based on achieved FN rates by running PMAO with 30 input weight vectors on each dataset of set A. Here each 15D feature vector $X$ consists of the input 5D weight vector for PMAO plus 10 features (used by~\cite{rubio2018characteristic}) extracted from the unaligned protein sequences. More details can be found in Appendix~\ref{appendix:train_xgboost} of the Supplementary file.
	\item Step 2: Fit an XGBoost regressor~\cite{chen2016xgboost}, a popular gradient tree boosting system, to the training data prepared in Step 1. For details please see Appendix~\ref{appendix:train_xgboost} in the Supplementary file.
\item Step 3: Calculate 100 well-spaced 5D weight vectors (Supplementary Figure~\ref{fig:100-weights}) using the method of~\cite{ref_dirs_energy}. Then prepare a ($100 \times 15$) test data, for each dataset under set B, by appending 10 extracted features as mentioned in Step 1 (\cite{rubio2018characteristic}) to each of the 100 weight vectors. This step uses 100 weight vectors rather than 30, as they offer a wider range of promising weight vectors as suggested by Supplementary Table~\ref{tab:pmao-100}.
	\item Step 4: Predict the FN rate for each test feature vector formed in Step 4 using the model obtained in Step 2.
	\item Step 5: For each dataset in set B, select the five test feature vectors having the most minor five predicted FN rates and output their respective weight vectors.
\end{itemize}

The five weight vectors detected in this way on four arbitrary datasets in set B are visualized in Figure~\ref{fig:some-good-weight-ml}. Supplementary Figure~\ref{fig:good-weight-ml} shows such weight vectors for all datasets. As a baseline, we randomly picked five weight vectors for each dataset under set B, from the 30 weight vectors of Figure~\ref{fig:weight}. In addition, we employ three criteria, based on simple rationales from MO optimization, for selecting three sets of five vectors as shown in Figure~\ref{fig:simple-weight}.
Thus we created five variants listed in Table~\ref{tab:variants-weight} by feeding into PMAO five different sets of five weight vectors to generate a tree-space containing five solutions rather than 30. We show the best and average FN rate achieved by these five variants, along with PASTA's FN rate, on each dataset under set B in Table~\ref{tab:pmao-5w-b}. On each dataset (i.e., row), the better (i.e., lower) FN rate values are marked with a darker shade. We find that the five best FN rate columns for five PMAO variants contain more dark shaded cells than PASTA. But the five average FN columns have fewer dark shaded cells than PASTA, indicating some lower quality solutions in the tree-space comprising five solutions. We conduct the Friedman test followed by Holm's post-hoc procedure to contrast PASTA and best FN rates of five PMAO variants with 95\% confidence. Table~\ref{tab:test-ml-weights} summarized the test results, which shows all the PMAO variants achieve lower ranks than PASTA, with PMAO-5C0 and PMAO-5DW being the top performers. Among all the variants, we find only PMAO-5C0 and PMAO-5DW to be significantly different from PASTA. Superimposing the best FN rates on Supplementary Figure~\ref{fig:good-weight-b} reveals that, PMAO-5C0 and PMAO-5DW can almost always capture the best or second-best or third-best of the overall PMAO best among its 30 solutions. 

Although PMAO-5C0 has a slightly lower Friedman rank than PMAO-5DW, please note that PMAO-5C0 uses the same set of weight vectors for all instances. On the other hand, PMAO-5DW offers the flexibility to learn a set of novel vectors that may best serve newly observed data based on its characteristics. As an example, let us consider the case in Figure~\ref{fig:unit-ml-weight} which depicts the five weight vectors detected through supervised learning for BB30008. We see that the vector leading to the best accuracy is completely biased toward the objective SIMNG which is contrary to the rationale behind PMAO-5C0. Although PMAO-5C1 can provide such a vector, it may not perform satisfactorily in general as we observe in our experiment.

%with five weight vectors detected by the regression model, we almost always capture the best or second best or third best of the overall PMAO best among its 30 solutions. \textbf{Although the random scheme (PMAO-5RW) also performs satisfactorily in this experiment, the significant difference between PMAO-5DW and PASTA, but not between PMAO-5RW and PASTA, highlights the potential of PMAO-5DW over PMAO-5RW. This finding answers our question affirmatively that a carefully designed machine learning approach can help us work with a few input weight vectors.}

\begin{figure}[!htbp]\begin{adjustwidth}{-0.6cm}{}
		\centering
	    \subfloat[BB11007]{\includegraphics[width=0.25\textwidth]{BB11007-good-weight}}
		\subfloat[BB20012]{\includegraphics[width=0.25\textwidth]{BB20012-good-weight}}\\
		\subfloat[BB30026]{\label{fig:unit-ml-weight} \includegraphics[width=0.25\textwidth]{BB30026-good-weight}}
		\subfloat[BB50009]{\includegraphics[width=0.25\textwidth]{BB50009-good-weight}}
	\end{adjustwidth}
	\caption{Visualization of five input weight vectors detected by our supervised regression based pipeline on four arbitrary datasets in set B. The y-axis portrays the weight values and the x-axis marks the achieved FN rate. The weight vectors are sorted in ascending order based on the achieved FN rate. }
	\label{fig:some-good-weight-ml}
\end{figure}

\begin{figure*}[pos=!htbp, align=\centering, width=17cm]
	%\begin{adjustwidth}{-0.6cm}{}
	
	\subfloat[PMAO-5C0]{\includegraphics[width=0.25\textwidth]{simple0}}
	\subfloat[PMAO-5C1]{\includegraphics[width=0.25\textwidth]{simple1}}
	\subfloat[PMAO-5C2]{\includegraphics[width=0.25\textwidth]{simple2}}
	%\end{adjustwidth}
	\caption{Visualization of five weight vectors selected by three different criteria. The y-axis portrays the weight values and the x-axis marks the position of the selected vector referred from Figure~\ref{fig:weight}. }
	\label{fig:simple-weight}
\end{figure*}

\begin{table*}[!htbp]
	\small
	\caption{PMAO variants based on selection of five input weight vectors.}
	\begin{tabular}{l|L{14cm}}
		Variant &  How to select the five input weight vectors for PMAO framework\\
		\hline
		PMAO-5DW  &  Detected from 100 well-spaced vectors based on lower FN rates predicted by a XGBoost regressor\\
		\hline
		PMAO-5RW  &  Randomly sampled from the 30 well-spaced weight vectors \\
		\hline
		PMAO-5C0  &  Five vectors from the 30 well-spaced weight vectors having the lowest five euclidean distance from $(0.2,0.2,0.2,0.2)^T$ (i.e., the diagonal weight vector). Rationale: well-balanced among the objectives. \\
		\hline
		PMAO-5C1  &  Five vectors from 30 well-spaced weight vectors having the highest five crowding distance~\cite{deb2002fast}. Rationale: well-scattered in the 5D unit simplex. \\
		\hline
		PMAO-5C2  &  Five vectors from 30 well-spaced weight vectors having the highest five crowding distance but excluding vectors having any value $> 0.7$. Rationale: well-scattered while avoiding bias towards any objective.\\
	\end{tabular}\label{tab:variants-weight}
\end{table*}




% Table generated by Excel2LaTeX from sheet 'stat-8I-30w-ML-simple-w'
\begin{table*}[pos=!htbp, align=\centering, width=15cm]
	\small
	\caption{Comparison of the five solutions generated by five PMAO variants of Table~\ref{tab:variants-weight} with respect to PASTA based on FN rate on set B datasets. For PMAO variants, we show the best and average FN rate of the five solutions. On each dataset (i.e., row), the better (i.e., lower) FN rate values are marked with a darker shade.}
	\begin{tabular}{|l|r|r|r|r|r|r|r|r|r|r|r|}
		\hline
		\multirow{2}{*}{Dataset} & \multirow{2}{*}{\makecell{PASTA\\FN rate}} & \multicolumn{2}{c|}{\makecell{PMO-5DW\\FN rate}} & \multicolumn{2}{c|}{\makecell{PMO-5RW\\FN rate}} & \multicolumn{2}{c|}{\makecell{PMO-5C0\\FN rate}} &  \multicolumn{2}{c|}{\makecell{PMO-5C1\\FN rate}} &  \multicolumn{2}{c|}{\makecell{PMO-5C2\\FN rate}} \\
		\cline{3-12}          &       & \multicolumn{1}{l|}{Best} & \multicolumn{1}{l|}{Avg} & \multicolumn{1}{l|}{Best} & \multicolumn{1}{l|}{Avg} & \multicolumn{1}{l|}{Best} & \multicolumn{1}{l|}{Avg} & \multicolumn{1}{l|}{Best} & \multicolumn{1}{l|}{Avg} & \multicolumn{1}{l|}{Best} & \multicolumn{1}{l|}{Avg} \\
		\hline
		BB11007 & \cellcolor[rgb]{ .686,  .871,  .733}0.50 & \cellcolor[rgb]{ .686,  .871,  .733}0.50 & \cellcolor[rgb]{ .988,  1,  .992}0.67 & \cellcolor[rgb]{ .384,  .745,  .478}0.33 & \cellcolor[rgb]{ .804,  .922,  .835}0.57 & \cellcolor[rgb]{ .686,  .871,  .733}0.50 & \cellcolor[rgb]{ .867,  .949,  .886}0.60 & \cellcolor[rgb]{ .384,  .745,  .478}0.33 & \cellcolor[rgb]{ .867,  .949,  .886}0.60 & \cellcolor[rgb]{ .686,  .871,  .733}0.50 & \cellcolor[rgb]{ .867,  .949,  .886}0.60 \\
		\hline
		BB11034 & \cellcolor[rgb]{ .635,  .851,  .69}0.40 & \cellcolor[rgb]{ .384,  .745,  .478}0.20 & \cellcolor[rgb]{ .584,  .827,  .647}0.36 & \cellcolor[rgb]{ .384,  .745,  .478}0.20 & \cellcolor[rgb]{ .988,  1,  .992}0.68 & \cellcolor[rgb]{ .384,  .745,  .478}0.20 & \cellcolor[rgb]{ .682,  .871,  .733}0.44 & \cellcolor[rgb]{ .635,  .851,  .69}0.40 & \cellcolor[rgb]{ .835,  .933,  .863}0.56 & \cellcolor[rgb]{ .384,  .745,  .478}0.20 & \cellcolor[rgb]{ .682,  .871,  .733}0.44 \\
		\hline
		BB11038 & \cellcolor[rgb]{ .686,  .871,  .733}0.40 & \cellcolor[rgb]{ .686,  .871,  .733}0.40 & \cellcolor[rgb]{ .867,  .945,  .886}0.64 & \cellcolor[rgb]{ .686,  .871,  .733}0.40 & \cellcolor[rgb]{ .776,  .91,  .812}0.52 & \cellcolor[rgb]{ .686,  .871,  .733}0.40 & \cellcolor[rgb]{ .776,  .91,  .812}0.52 & \cellcolor[rgb]{ .384,  .745,  .478}0.00 & \cellcolor[rgb]{ .745,  .898,  .784}0.48 & \cellcolor[rgb]{ .988,  1,  .992}0.80 & \cellcolor[rgb]{ .988,  1,  .992}0.80 \\
		\hline
		BB11019 & \cellcolor[rgb]{ .886,  .957,  .906}0.29 & \cellcolor[rgb]{ .886,  .957,  .906}0.29 & \cellcolor[rgb]{ .886,  .957,  .906}0.29 & \cellcolor[rgb]{ .886,  .957,  .906}0.29 & \cellcolor[rgb]{ .988,  1,  .992}0.31 & \cellcolor[rgb]{ .384,  .745,  .478}0.14 & \cellcolor[rgb]{ .886,  .957,  .906}0.29 & \cellcolor[rgb]{ .886,  .957,  .906}0.29 & \cellcolor[rgb]{ .886,  .957,  .906}0.29 & \cellcolor[rgb]{ .886,  .957,  .906}0.29 & \cellcolor[rgb]{ .988,  1,  .992}0.31 \\
		\hline
		BB12005 & \cellcolor[rgb]{ .384,  .745,  .478}0.33 & \cellcolor[rgb]{ .384,  .745,  .478}0.33 & \cellcolor[rgb]{ .988,  1,  .992}0.37 & \cellcolor[rgb]{ .384,  .745,  .478}0.33 & \cellcolor[rgb]{ .988,  1,  .992}0.37 & \cellcolor[rgb]{ .384,  .745,  .478}0.33 & \cellcolor[rgb]{ .384,  .745,  .478}0.33 & \cellcolor[rgb]{ .384,  .745,  .478}0.33 & \cellcolor[rgb]{ .384,  .745,  .478}0.33 & \cellcolor[rgb]{ .384,  .745,  .478}0.33 & \cellcolor[rgb]{ .384,  .745,  .478}0.33 \\
		\hline
		BB12029 & \cellcolor[rgb]{ .886,  .957,  .906}0.44 & \cellcolor[rgb]{ .384,  .745,  .478}0.33 & \cellcolor[rgb]{ .686,  .871,  .733}0.40 & \cellcolor[rgb]{ .886,  .957,  .906}0.44 & \cellcolor[rgb]{ .886,  .957,  .906}0.44 & \cellcolor[rgb]{ .384,  .745,  .478}0.33 & \cellcolor[rgb]{ .784,  .914,  .82}0.42 & \cellcolor[rgb]{ .384,  .745,  .478}0.33 & \cellcolor[rgb]{ .784,  .914,  .82}0.42 & \cellcolor[rgb]{ .886,  .957,  .906}0.44 & \cellcolor[rgb]{ .988,  1,  .992}0.47 \\
		\hline
		BB12026 & \cellcolor[rgb]{ .816,  .925,  .843}0.53 & \cellcolor[rgb]{ .671,  .863,  .722}0.47 & \cellcolor[rgb]{ .816,  .925,  .843}0.53 & \cellcolor[rgb]{ .816,  .925,  .843}0.53 & \cellcolor[rgb]{ .957,  .984,  .965}0.60 & \cellcolor[rgb]{ .671,  .863,  .722}0.47 & \cellcolor[rgb]{ .988,  1,  .992}0.61 & \cellcolor[rgb]{ .384,  .745,  .478}0.33 & \cellcolor[rgb]{ .784,  .914,  .82}0.52 & \cellcolor[rgb]{ .671,  .863,  .722}0.47 & \cellcolor[rgb]{ .843,  .937,  .867}0.55 \\
		\hline
		BB12037 & \cellcolor[rgb]{ .384,  .745,  .478}0.40 & \cellcolor[rgb]{ .914,  .969,  .929}0.70 & \cellcolor[rgb]{ .988,  1,  .992}0.74 & \cellcolor[rgb]{ .384,  .745,  .478}0.40 & \cellcolor[rgb]{ .737,  .894,  .78}0.60 & \cellcolor[rgb]{ .561,  .82,  .627}0.50 & \cellcolor[rgb]{ .737,  .894,  .78}0.60 & \cellcolor[rgb]{ .384,  .745,  .478}0.40 & \cellcolor[rgb]{ .843,  .937,  .871}0.66 & \cellcolor[rgb]{ .561,  .82,  .627}0.50 & \cellcolor[rgb]{ .843,  .937,  .871}0.66 \\
		\hline
		BB20002 & \cellcolor[rgb]{ .533,  .808,  .604}0.65 & \cellcolor[rgb]{ .384,  .745,  .478}0.59 & \cellcolor[rgb]{ .624,  .847,  .682}0.68 & \cellcolor[rgb]{ .384,  .745,  .478}0.59 & \cellcolor[rgb]{ .925,  .973,  .937}0.80 & \cellcolor[rgb]{ .384,  .745,  .478}0.59 & \cellcolor[rgb]{ .686,  .871,  .733}0.71 & \cellcolor[rgb]{ .533,  .808,  .604}0.65 & \cellcolor[rgb]{ .776,  .91,  .812}0.74 & \cellcolor[rgb]{ .533,  .808,  .604}0.65 & \cellcolor[rgb]{ .988,  1,  .992}0.82 \\
		\hline
		BB20012 & \cellcolor[rgb]{ .941,  .98,  .953}0.29 & \cellcolor[rgb]{ .608,  .839,  .667}0.17 & \cellcolor[rgb]{ .831,  .933,  .859}0.25 & \cellcolor[rgb]{ .718,  .886,  .761}0.21 & \cellcolor[rgb]{ .851,  .941,  .875}0.26 & \cellcolor[rgb]{ .608,  .839,  .667}0.17 & \cellcolor[rgb]{ .761,  .902,  .8}0.23 & \cellcolor[rgb]{ .831,  .933,  .859}0.25 & \cellcolor[rgb]{ .988,  1,  .992}0.31 & \cellcolor[rgb]{ .384,  .745,  .478}0.08 & \cellcolor[rgb]{ .831,  .933,  .859}0.25 \\
		\hline
		BB20030 & \cellcolor[rgb]{ .384,  .745,  .478}0.64 & \cellcolor[rgb]{ .384,  .745,  .478}0.64 & \cellcolor[rgb]{ .686,  .871,  .733}0.69 & \cellcolor[rgb]{ .384,  .745,  .478}0.64 & \cellcolor[rgb]{ .659,  .859,  .71}0.68 & \cellcolor[rgb]{ .384,  .745,  .478}0.64 & \cellcolor[rgb]{ .875,  .953,  .898}0.72 & \cellcolor[rgb]{ .796,  .918,  .827}0.70 & \cellcolor[rgb]{ .988,  1,  .992}0.74 & \cellcolor[rgb]{ .659,  .859,  .71}0.68 & \cellcolor[rgb]{ .961,  .988,  .969}0.73 \\
		\hline
		BB20037 & \cellcolor[rgb]{ .384,  .745,  .478}0.13 & \cellcolor[rgb]{ .537,  .808,  .608}0.16 & \cellcolor[rgb]{ .769,  .906,  .804}0.21 & \cellcolor[rgb]{ .847,  .937,  .871}0.23 & \cellcolor[rgb]{ .941,  .976,  .949}0.25 & \cellcolor[rgb]{ .847,  .941,  .871}0.23 & \cellcolor[rgb]{ .988,  1,  .992}0.25 & \cellcolor[rgb]{ .384,  .745,  .478}0.13 & \cellcolor[rgb]{ .769,  .906,  .804}0.21 & \cellcolor[rgb]{ .537,  .808,  .608}0.16 & \cellcolor[rgb]{ .722,  .886,  .765}0.20 \\
		\hline
		BB30003 & \cellcolor[rgb]{ .929,  .976,  .945}0.30 & \cellcolor[rgb]{ .655,  .859,  .71}0.28 & \cellcolor[rgb]{ .804,  .922,  .835}0.29 & \cellcolor[rgb]{ .749,  .898,  .788}0.29 & \cellcolor[rgb]{ .914,  .969,  .929}0.30 & \cellcolor[rgb]{ .565,  .82,  .631}0.27 & \cellcolor[rgb]{ .839,  .937,  .867}0.29 & \cellcolor[rgb]{ .749,  .898,  .788}0.29 & \cellcolor[rgb]{ .988,  1,  .992}0.31 & \cellcolor[rgb]{ .384,  .745,  .478}0.26 & \cellcolor[rgb]{ .729,  .89,  .773}0.29 \\
		\hline
		BB30021 & \cellcolor[rgb]{ .8,  .922,  .831}0.36 & \cellcolor[rgb]{ .384,  .745,  .478}0.32 & \cellcolor[rgb]{ .835,  .933,  .863}0.36 & \cellcolor[rgb]{ .384,  .745,  .478}0.32 & \cellcolor[rgb]{ .686,  .871,  .733}0.35 & \cellcolor[rgb]{ .384,  .745,  .478}0.32 & \cellcolor[rgb]{ .82,  .925,  .847}0.36 & \cellcolor[rgb]{ .718,  .886,  .761}0.35 & \cellcolor[rgb]{ .988,  1,  .992}0.37 & \cellcolor[rgb]{ .384,  .745,  .478}0.32 & \cellcolor[rgb]{ .851,  .941,  .875}0.36 \\
		\hline
		BB30026 & \cellcolor[rgb]{ .812,  .925,  .843}0.19 & \cellcolor[rgb]{ .384,  .745,  .478}0.15 & \cellcolor[rgb]{ .871,  .949,  .894}0.20 & \cellcolor[rgb]{ .671,  .863,  .722}0.18 & \cellcolor[rgb]{ .929,  .973,  .941}0.20 & \cellcolor[rgb]{ .671,  .863,  .722}0.18 & \cellcolor[rgb]{ .988,  1,  .992}0.21 & \cellcolor[rgb]{ .671,  .863,  .722}0.18 & \cellcolor[rgb]{ .957,  .984,  .965}0.21 & \cellcolor[rgb]{ .525,  .804,  .6}0.17 & \cellcolor[rgb]{ .929,  .973,  .941}0.20 \\
		\hline
		BB30011 & \cellcolor[rgb]{ .988,  1,  .992}0.35 & \cellcolor[rgb]{ .584,  .827,  .647}0.32 & \cellcolor[rgb]{ .867,  .949,  .886}0.35 & \cellcolor[rgb]{ .988,  1,  .992}0.35 & \cellcolor[rgb]{ .988,  1,  .992}0.35 & \cellcolor[rgb]{ .584,  .827,  .647}0.32 & \cellcolor[rgb]{ .906,  .965,  .922}0.35 & \cellcolor[rgb]{ .384,  .745,  .478}0.31 & \cellcolor[rgb]{ .784,  .914,  .82}0.34 & \cellcolor[rgb]{ .584,  .827,  .647}0.32 & \cellcolor[rgb]{ .824,  .929,  .855}0.34 \\
		\hline
		BB40009 & \cellcolor[rgb]{ .384,  .745,  .478}0.19 & \cellcolor[rgb]{ .384,  .745,  .478}0.19 & \cellcolor[rgb]{ .533,  .808,  .604}0.20 & \cellcolor[rgb]{ .384,  .745,  .478}0.19 & \cellcolor[rgb]{ .988,  1,  .992}0.24 & \cellcolor[rgb]{ .384,  .745,  .478}0.19 & \cellcolor[rgb]{ .988,  1,  .992}0.24 & \cellcolor[rgb]{ .384,  .745,  .478}0.19 & \cellcolor[rgb]{ .686,  .871,  .733}0.21 & \cellcolor[rgb]{ .384,  .745,  .478}0.19 & \cellcolor[rgb]{ .835,  .933,  .863}0.23 \\
		\hline
		BB40019 & \cellcolor[rgb]{ .988,  1,  .992}0.43 & \cellcolor[rgb]{ .384,  .745,  .478}0.29 & \cellcolor[rgb]{ .745,  .898,  .784}0.37 & \cellcolor[rgb]{ .384,  .745,  .478}0.29 & \cellcolor[rgb]{ .624,  .847,  .682}0.34 & \cellcolor[rgb]{ .384,  .745,  .478}0.29 & \cellcolor[rgb]{ .624,  .847,  .682}0.34 & \cellcolor[rgb]{ .384,  .745,  .478}0.29 & \cellcolor[rgb]{ .624,  .847,  .682}0.34 & \cellcolor[rgb]{ .384,  .745,  .478}0.29 & \cellcolor[rgb]{ .745,  .898,  .784}0.37 \\
		\hline
		BB40033 & \cellcolor[rgb]{ .988,  1,  .992}0.13 & \cellcolor[rgb]{ .384,  .745,  .478}0.06 & \cellcolor[rgb]{ .867,  .949,  .886}0.11 & \cellcolor[rgb]{ .384,  .745,  .478}0.06 & \cellcolor[rgb]{ .867,  .949,  .886}0.11 & \cellcolor[rgb]{ .384,  .745,  .478}0.06 & \cellcolor[rgb]{ .502,  .792,  .58}0.08 & \cellcolor[rgb]{ .384,  .745,  .478}0.06 & \cellcolor[rgb]{ .867,  .949,  .886}0.11 & \cellcolor[rgb]{ .384,  .745,  .478}0.06 & \cellcolor[rgb]{ .867,  .949,  .886}0.11 \\
		\hline
		BB40006 & \cellcolor[rgb]{ .384,  .745,  .478}0.00 & \cellcolor[rgb]{ .384,  .745,  .478}0.00 & \cellcolor[rgb]{ .718,  .886,  .761}0.09 & \cellcolor[rgb]{ .718,  .886,  .761}0.09 & \cellcolor[rgb]{ .784,  .914,  .82}0.11 & \cellcolor[rgb]{ .718,  .886,  .761}0.09 & \cellcolor[rgb]{ .988,  1,  .992}0.16 & \cellcolor[rgb]{ .384,  .745,  .478}0.00 & \cellcolor[rgb]{ .784,  .914,  .82}0.11 & \cellcolor[rgb]{ .384,  .745,  .478}0.00 & \cellcolor[rgb]{ .784,  .914,  .82}0.11 \\
		\hline
		BB50002 & \cellcolor[rgb]{ .584,  .827,  .647}0.50 & \cellcolor[rgb]{ .384,  .745,  .478}0.40 & \cellcolor[rgb]{ .663,  .863,  .718}0.54 & \cellcolor[rgb]{ .584,  .827,  .647}0.50 & \cellcolor[rgb]{ .745,  .898,  .784}0.58 & \cellcolor[rgb]{ .384,  .745,  .478}0.40 & \cellcolor[rgb]{ .988,  1,  .992}0.70 & \cellcolor[rgb]{ .584,  .827,  .647}0.50 & \cellcolor[rgb]{ .745,  .898,  .784}0.58 & \cellcolor[rgb]{ .584,  .827,  .647}0.50 & \cellcolor[rgb]{ .745,  .898,  .784}0.58 \\
		\hline
		BB50009 & \cellcolor[rgb]{ .929,  .976,  .945}0.24 & \cellcolor[rgb]{ .796,  .918,  .827}0.20 & \cellcolor[rgb]{ .906,  .965,  .922}0.23 & \cellcolor[rgb]{ .384,  .745,  .478}0.08 & \cellcolor[rgb]{ .796,  .918,  .827}0.20 & \cellcolor[rgb]{ .796,  .918,  .827}0.20 & \cellcolor[rgb]{ .988,  1,  .992}0.26 & \cellcolor[rgb]{ .659,  .859,  .71}0.16 & \cellcolor[rgb]{ .902,  .965,  .922}0.23 & \cellcolor[rgb]{ .659,  .859,  .71}0.16 & \cellcolor[rgb]{ .929,  .976,  .945}0.24 \\
		\hline
		BB50014 & \cellcolor[rgb]{ .867,  .945,  .886}0.41 & \cellcolor[rgb]{ .867,  .945,  .886}0.41 & \cellcolor[rgb]{ .961,  .988,  .969}0.44 & \cellcolor[rgb]{ .988,  1,  .992}0.44 & \cellcolor[rgb]{ .988,  1,  .992}0.44 & \cellcolor[rgb]{ .502,  .792,  .58}0.30 & \cellcolor[rgb]{ .867,  .945,  .886}0.41 & \cellcolor[rgb]{ .502,  .796,  .58}0.30 & \cellcolor[rgb]{ .89,  .957,  .91}0.41 & \cellcolor[rgb]{ .384,  .745,  .478}0.26 & \cellcolor[rgb]{ .843,  .937,  .867}0.40 \\
		\hline
		BB50006 & \cellcolor[rgb]{ .769,  .906,  .804}0.21 & \cellcolor[rgb]{ .459,  .776,  .541}0.14 & \cellcolor[rgb]{ .847,  .937,  .871}0.23 & \cellcolor[rgb]{ .69,  .875,  .741}0.19 & \cellcolor[rgb]{ .973,  .992,  .976}0.26 & \cellcolor[rgb]{ .384,  .745,  .478}0.12 & \cellcolor[rgb]{ .753,  .898,  .792}0.21 & \cellcolor[rgb]{ .537,  .808,  .608}0.16 & \cellcolor[rgb]{ .831,  .933,  .859}0.22 & \cellcolor[rgb]{ .69,  .875,  .741}0.19 & \cellcolor[rgb]{ .988,  1,  .992}0.26 \\
		\hline
	\end{tabular}%
	\label{tab:pmao-5w-b}%
\end{table*}%

% Table generated by Excel2LaTeX from sheet 'stat test'
\begin{table*}[pos=!htbp, align=\centering, width=17cm]
	\caption{\underline{The Friedman test (Column 2):} Friedman ranks (lower is better) of PASTA and the five PMAO variants of Table~\ref{tab:variants-weight} based on the best FN rates reported in Table~\ref{tab:pmao-5w-b}. We also show the computed statistics and corresponding $ p $-value.
		\underline{Holm's post-hoc procedure (Columns 3 - 8):} Comparison among PASTA and two PMAO variants using the Holm's post-hoc procedures. Each entry shows the adjusted $p$-value which indicates the significance of the difference in performance between two methods. The significant differences are marked with green color.}
	\begin{tabular}{|l|r||cccccc|}
		\hline
		\multicolumn{1}{|c|}{1} & \multicolumn{1}{c||}{2} & \multicolumn{1}{c|}{3} & \multicolumn{1}{c|}{4} & \multicolumn{1}{c|}{5} & \multicolumn{1}{c|}{6} & \multicolumn{1}{c|}{7} & 8 \\
		\hline
		\multicolumn{1}{|c|}{\multirow{2}[4]{*}{Method}} & \multirow{2}{*}{\makecell{Friedman\\Rank*}} & \multicolumn{6}{c|}{Holm's adjusted $p$-value} \\
		\cline{3-8}          &       & \multicolumn{1}{l|}{PMAO-5C0} & \multicolumn{1}{l|}{PMAO-5DW} & \multicolumn{1}{l|}{PMAO-5C1} & \multicolumn{1}{l|}{PMAO-5RW} & \multicolumn{1}{l|}{PMAO-5C2} & \multicolumn{1}{l|}{PASTA} \\
		\hline
		PMAO-5C0 & 2.8958 & \multicolumn{1}{c|}{-} & \multicolumn{1}{r|}{1.0000} & \multicolumn{1}{r|}{1.0000} & \multicolumn{1}{r|}{1.0000} & \multicolumn{1}{r|}{1.0000} & \multicolumn{1}{r|}{\cellcolor[rgb]{ 0,  .69,  .314}0.0304} \\
		\hline
		PMAO-5DW & 2.9583 & \multicolumn{1}{r|}{1.0000} & \multicolumn{1}{c|}{-} & \multicolumn{1}{r|}{1.0000} & \multicolumn{1}{r|}{1.0000} & \multicolumn{1}{r|}{1.0000} & \multicolumn{1}{r|}{\cellcolor[rgb]{ 0,  .69,  .314}0.0417} \\
		\hline
		PMAO-5C1 & 3.1250 & \multicolumn{1}{r|}{1.0000} & \multicolumn{1}{r|}{1.0000} & \multicolumn{1}{c|}{-} & \multicolumn{1}{r|}{1.0000} & \multicolumn{1}{r|}{1.0000} & \multicolumn{1}{r|}{0.1011} \\
		\hline
		PMAO-5RW  & 3.6667 & \multicolumn{1}{r|}{1.0000} & \multicolumn{1}{r|}{1.0000} & \multicolumn{1}{r|}{1.0000} & \multicolumn{1}{c|}{-} & \multicolumn{1}{r|}{1.0000} & \multicolumn{1}{r|}{1.0000} \\
		\hline
		 PMAO-5C2 & 3.7917 & \multicolumn{1}{r|}{1.0000} & \multicolumn{1}{r|}{1.0000} & \multicolumn{1}{r|}{1.0000} & \multicolumn{1}{r|}{1.0000} & \multicolumn{1}{c|}{-} & \multicolumn{1}{r|}{1.0000} \\
		\hline
		PASTA & 4.5625 & \multicolumn{1}{r|}{\cellcolor[rgb]{ 0,  .69,  .314}0.0304} & \multicolumn{1}{r|}{\cellcolor[rgb]{ 0,  .69,  .314}0.0417} & \multicolumn{1}{r|}{0.1011} & \multicolumn{1}{r|}{1.0000} & \multicolumn{1}{r|}{1.0000} & - \\
		\hline \hline
		*Statistic &    3.0363   & \multicolumn{6}{c|}{\multirow{2}[4]{*}{N/A}} \\
		\cline{1-2}    *$p$-value &  0.0130     & \multicolumn{6}{c|}{} \\
		\hline
	\end{tabular}%
	\label{tab:test-ml-weights}%
\end{table*}%



\subsubsection{Obtain one solution from PMAO-5DW}
In pursuit of a single high-quality solution, we proceed even further. We employ the following two schemes to obtain a single high-quality solution by summarizing the five solutions generated by PMAO-5DW.
\begin{enumerate}
\item GRED: Summarize the five candidate trees using greedy consensus method of PAUP* (Phylogenetic Analysis Using PAUP)\footnote{https://paup.phylosolutions.com/}.
	\item AST: Summarize the five candidate trees based on quartet consistency using ASTRAL~\cite{zhang2018astral}, one of the most accurate and widely used coalescent-based methods to infer species trees.
\end{enumerate}
We compare the FN rates of the aforementioned summarizing schemes with PASTA on all BAliBASE 3.0 datasets, except set A to eliminate any bias due to their involvement in training the regression model, in Supplementary Table~\ref{tab:obtain-single-ml}. Then we apply the Friedman Aligned Ranks test followed by Holm's post-hoc procedure on these data and report the results in Table~\ref{tab:test-summary}. We observe that GRED and AST schemes exhibit significantly better performance than PASTA, with GRED being the top performer. However, these schemes are not always able to obtain the best FN rates of PMAO-5DW, and thus we see that PASTA performs better than GRED and AST (Supplementary Table~\ref{tab:obtain-single-ml}) in 37 cases out of 191. This suggests that, although promising, further refinement of this approach or application of other strategies may be in order.

\begin{table}[htbp]
  \small
  \caption{\underline{Friedman Aligned Ranks test (Column 2):} Friedman Aligned ranks (lower is better) of PASTA and two schemes for summarizing five solutions generated by PMAO-5DW based on FN rates reported in Supplementary Table~\ref{tab:obtain-single-ml}. We also show the computed statistics and corresponding $ p $-value.
	\underline{Holm's post-hoc procedure (Columns 3 - 5):} Comparison among PASTA and two summarizing schemes using the Holm's post-hoc procedures. Each entry shows the adjusted $p$-value which indicates the significance of the difference in performance between two methods. The significant differences are marked with green color.}
    \begin{tabular}{|l|r||ccc|}
    \hline
    \multicolumn{1}{|c|}{1} & \multicolumn{1}{c||}{2} & \multicolumn{1}{c|}{3} & \multicolumn{1}{c|}{4} & 5 \\
    \hline
	\multirow{2}{*}{\makecell{Methods}} & \multirow{2}{*}{\makecell{Friedman\\Aligned rank*}} & \multicolumn{3}{c|}{Holm's adjusted $p$-value} \\
\cline{3-5}          &       & \multicolumn{1}{l|}{GRED} & \multicolumn{1}{l|}{AST} & \multicolumn{1}{l|}{PASTA} \\
    \hline
    GRED  & 263.9922 & \multicolumn{1}{c|}{-} & \multicolumn{1}{r|}{0.5982} & \multicolumn{1}{r|}{\cellcolor[rgb]{ 0,  .69,  .314}0.0012} \\
    \hline
    AST   & 272.9189 & \multicolumn{1}{r|}{0.5982} & \multicolumn{1}{c|}{-} & \multicolumn{1}{r|}{\cellcolor[rgb]{ 0,  .69,  .314}0.0051} \\
    \hline
    PASTA & 324.0890 & \multicolumn{1}{r|}{\cellcolor[rgb]{ 0,  .69,  .314}0.0012} & \multicolumn{1}{r|}{\cellcolor[rgb]{ 0,  .69,  .314}0.0051} & - \\
    \hline \hline
    *Statistic & 10.7149 & \multicolumn{3}{c|}{\multirow{2}{*}{N/A}} \\
\cline{1-2}    *$p$-value & 0.0047 & \multicolumn{3}{c|}{} \\
    \hline
    \end{tabular}\label{tab:test-summary}\end{table} 

\subsubsection{Obtain one solution from PMAO-5C0}
As PMAO-5C0 seems to be promising based on experimentation summarized in Table~\ref{tab:pmao-5w-b} and Table~\ref{tab:test-ml-weights}, we summarize the five solutions generated by PMAO-5C0 using the aforementioned schemes on all BAliBASE 3.0 datasets. We compare their performance with PASTA in Supplementary Table~\ref{tab:obtain-single-simple} and conduct Friedman Aligned Ranks based on that data reported in Table~\ref{tab:test-summary-simple}. Similar to the PMAO-5DW case, we find that both summarizing schemes exhibit significantly better performance than PASTA and on average GRED is better than AST. And unlike original PMAO solutions, we find that the PASTA solution is better than GRED and AST in 39 cases out of 218.



\begin{table}[!htbp]
	\small
	\caption{\underline{Friedman Aligned Ranks test (Column 2):} Friedman Aligned ranks (lower is better) of PASTA and two schemes for summarizing five solutions generated by PMAO-5C0 based on FN rates reported in Supplementary Table~\ref{tab:obtain-single-simple}. We also show the computed statistics and corresponding $ p $-value.
		\underline{Holm's post-hoc procedure (Columns 3 - 5):} Comparison among PASTA and two summarizing schemes using the Holm's post-hoc procedures. Each entry shows the adjusted $p$-value which indicates the significance of the difference in performance between two methods. The significant differences are marked with green color.}
	\begin{tabular}{|l|r||ccc|}
		\hline
		\multicolumn{1}{|c|}{1} & \multicolumn{1}{c||}{2} & \multicolumn{1}{c|}{3} & \multicolumn{1}{c|}{4} & 5 \\
		\hline
		\multirow{2}{*}{\makecell{Methods}} & \multirow{2}{*}{\makecell{Friedman\\Aligned rank*}} & \multicolumn{3}{c|}{Holm's adjusted $p$-value} \\
		\cline{3-5}          &       & \multicolumn{1}{l|}{GRED} & \multicolumn{1}{l|}{AST} & \multicolumn{1}{l|}{PASTA} \\
		\hline
		GRED  & 293.4839 & \multicolumn{1}{c|}{-} & \multicolumn{1}{r|}{0.3330} & \multicolumn{1}{r|}{\cellcolor[rgb]{ 0,  .69,  .314}0.0000} \\
		\hline
		AST   & 311.0046 & \multicolumn{1}{r|}{0.3330} & \multicolumn{1}{c|}{-} & \multicolumn{1}{r|}{\cellcolor[rgb]{ 0,  .69,  .314}0.0004} \\
		\hline
		PASTA & 378.0115 & \multicolumn{1}{r|}{\cellcolor[rgb]{ 0,  .69,  .314}0.0000} & \multicolumn{1}{r|}{\cellcolor[rgb]{ 0,  .69,  .314}0.0004} & - \\
		\hline \hline
		*Statistic & 17.7816 & \multicolumn{3}{c|}{\multirow{2}{*}{N/A}} \\
		\cline{1-2}    *$p$-value & 0.0001 & \multicolumn{3}{c|}{} \\
		\hline
	\end{tabular}\label{tab:test-summary-simple}\end{table} 

\section{Conclusion}
This paper focused on inferring better phylogenetic trees from MSAs by incorporating many application-aware objective functions through decomposition-based MO principles. In this direction, we developed the PMAO framework, which is based on PASTA, one of the most celebrated algorithms/tools in this regard. We evaluated the PMAO framework and examined its capability to yield high-quality tress by experimenting on the widely used BAliBASE 3.0 benchmark. The PMAO framework, like other MO algorithms, outputs a good number of alternative solutions that are equivalent (usually referred to as the non-dominated solutions) in the context of the conflicting objectives considered in the MO framework. Some of these solutions may however be of relatively lower quality from the application perspective. To this end, we innovatively employ machine learning to help PMAO generate only five solutions encompassing at least one top-quality solution. Furthermore, summarizing those five solutions to obtain a single one can offer better accuracy over PASTA in most of the cases, although not as good as the overall PMAO best. A possible reason might be the presence of some lower quality trees in the PMAO output and the fact that our employed summarizing methods treat each input with equal importance. This it seems that some sort of weighted summarizing scheme might perform better. As a future extension, we endeavor to enhance the summarizing approach by weighting the input trees by a likelihood score and to apply other strategies of leveraging the alternative MSAs/trees to single out the overall PMAO best solution.

Notably, in this work, we primarily focused on improving the accuracy of PASTA and hence designed our experiments accordingly. We did not experiment with scalability as PASTA is considered to be highly scalable. PMAO doesn’t hamper that scalability in any way, rather it leverages the high-scalability of PASTA. To elaborate, PMAO just adds the computation of four simple objectives within PASTA (which can be computed in parallel). In addition to that, PMAO just introduces a constant factor (i.e., number of weight vectors) to the runtime of PASTA (which also can be ignored if we can execute different weight vectors in parallel). Therefore, theoretically, PMAO only adds a constant factor of additional complexity keeping the framework as scalable as PASTA. %To verify our claim, we executed PASTA and PMAO on BB20041 (number of sequences: 48, average sequences length: 695) using the same experimental settings. PASTA took 56.7 seconds and PMAO took 73.1 seconds on average for a single weight vector.  

We believe this work will further encourage researchers to investigate various application-aware measures for computing and evaluating MSAs. Such effort will potentially prompt more experimental studies addressing specific application domains and ultimately will propel our understanding of MSAs and their impact in various domains in bioinformatics, i.e., phylogeny estimation, protein structure, and function prediction, orthology prediction, etc. Consequently, we expect to see new scalable MSA tools by simultaneously optimizing multiple appropriate optimization criteria.
 
\section*{Competing interests}
There is NO Competing Interest.


\section*{Acknowledgments}
The first author is supported by the ICT Doctoral Fellowship administered by ICT Division, Government of People’s Republic of Bangladesh.



\bibliographystyle{plain}
\bibliography{mybibfile}

\vskip12pt

\bio{}
\textbf{Muhammad Ali Nayeem} is a Ph.D. researcher at the Department of Computer Science and Engineering (CSE), Bangladesh University of Engineering and Technology (BUET), Dhaka, Bangladesh. His research focuses on the application of computational intelligence to tackle real-life optimization problems.
\endbio

\bio{}
\textbf{Md. Shamsuzzoha Bayzid} is an Associate Professor at the Department of CSE, BUET. The overarching goal of Dr. Bayzid’s research is to answer impactful biological questions, especially those related to the study of evolution, by developing algorithms that can accurately analyze very large genome-scale datasets.
\endbio

\bio{}
\textbf{Naser Anjum Samudro} received his B.Sc (Eng.) degree from the Department of CSE at BUET. He is working as a Software Engineer (L2) at Chaldal Tech, Bangladesh.
\endbio

\bio{}
\textbf{Mohammad Saifur Rahman} is an Associate Professor at the Department of CSE, BUET. His research interest included Bioinformatics, Networking, Distributed Systems, Algorithms.
\endbio

\bio{}
\textbf{M. Sohel Rahman} is a Professor of the Department of CSE, BUET. He worked as a Visiting Research Fellow of King’s College London, UK during 2008-2011 and as a Visiting Senior Research Fellow there during 2014-15. He is a Senior Member of both IEEE and ACM; member of American Mathematical Society (AMS) and London Mathematical Society (LMS). 
\endbio
\end{document}

\begin{comment}
\begin{table}[!htbp]
\small
\caption{Comparison of the five solutions generated by two PMAO variants with respect to PASTA based on FN rate on datasets under set B. For PMAO, we show the best and average FN rate of the five solutions. The better values are marked with darker shade.}
\begin{tabular}{|l|r|r|r|r|r|}
\hline
\multirow{2}{*}{Dataset} & \multirow{2}{*}{\makecell{PASTA\\FN rate}} & \multicolumn{2}{c|}{\makecell{PMO-5DW\\FN rate}} & \multicolumn{2}{c|}{\makecell{PMO-5RW\\FN rate}} \\
\cline{3-6}          &       & Best & Avg & Best & Avg \\
\hline
BB11007 & \cellcolor[rgb]{ .686,  .871,  .733}0.50 & \cellcolor[rgb]{ .686,  .871,  .733}0.50 & \cellcolor[rgb]{ .988,  1,  .992}0.67 & \cellcolor[rgb]{ .384,  .745,  .478}0.33 & \cellcolor[rgb]{ .804,  .922,  .835}0.57 \\
\hline
BB11034 & \cellcolor[rgb]{ .635,  .851,  .69}0.40 & \cellcolor[rgb]{ .384,  .745,  .478}0.20 & \cellcolor[rgb]{ .584,  .827,  .647}0.36 & \cellcolor[rgb]{ .384,  .745,  .478}0.20 & \cellcolor[rgb]{ .988,  1,  .992}0.68 \\
\hline
BB11038 & \cellcolor[rgb]{ .384,  .745,  .478}0.40 & \cellcolor[rgb]{ .384,  .745,  .478}0.40 & \cellcolor[rgb]{ .988,  1,  .992}0.64 & \cellcolor[rgb]{ .384,  .745,  .478}0.40 & \cellcolor[rgb]{ .686,  .871,  .733}0.52 \\
\hline
BB11019 & \cellcolor[rgb]{ .384,  .745,  .478}0.29 & \cellcolor[rgb]{ .384,  .745,  .478}0.29 & \cellcolor[rgb]{ .384,  .745,  .478}0.29 & \cellcolor[rgb]{ .384,  .745,  .478}0.29 & \cellcolor[rgb]{ .988,  1,  .992}0.31 \\
\hline
BB12005 & \cellcolor[rgb]{ .384,  .745,  .478}0.33 & \cellcolor[rgb]{ .384,  .745,  .478}0.33 & \cellcolor[rgb]{ .988,  1,  .992}0.37 & \cellcolor[rgb]{ .384,  .745,  .478}0.33 & \cellcolor[rgb]{ .988,  1,  .992}0.37 \\
\hline
BB12029 & \cellcolor[rgb]{ .988,  1,  .992}0.44 & \cellcolor[rgb]{ .384,  .745,  .478}0.33 & \cellcolor[rgb]{ .745,  .894,  .784}0.40 & \cellcolor[rgb]{ .988,  1,  .992}0.44 & \cellcolor[rgb]{ .988,  1,  .992}0.44 \\
\hline
BB12026 & \cellcolor[rgb]{ .682,  .871,  .733}0.53 & \cellcolor[rgb]{ .384,  .745,  .478}0.47 & \cellcolor[rgb]{ .682,  .871,  .733}0.53 & \cellcolor[rgb]{ .682,  .871,  .733}0.53 & \cellcolor[rgb]{ .988,  1,  .992}0.60 \\
\hline
BB12037 & \cellcolor[rgb]{ .384,  .745,  .478}0.40 & \cellcolor[rgb]{ .914,  .969,  .929}0.70 & \cellcolor[rgb]{ .988,  1,  .992}0.74 & \cellcolor[rgb]{ .384,  .745,  .478}0.40 & \cellcolor[rgb]{ .737,  .894,  .78}0.60 \\
\hline
BB20002 & \cellcolor[rgb]{ .549,  .816,  .62}0.65 & \cellcolor[rgb]{ .384,  .745,  .478}0.59 & \cellcolor[rgb]{ .651,  .855,  .706}0.68 & \cellcolor[rgb]{ .384,  .745,  .478}0.59 & \cellcolor[rgb]{ .988,  1,  .992}0.80 \\
\hline
BB20012 & \cellcolor[rgb]{ .988,  1,  .992}0.29 & \cellcolor[rgb]{ .384,  .745,  .478}0.17 & \cellcolor[rgb]{ .784,  .914,  .82}0.25 & \cellcolor[rgb]{ .584,  .827,  .647}0.21 & \cellcolor[rgb]{ .824,  .929,  .855}0.26 \\
\hline
BB20030 & \cellcolor[rgb]{ .384,  .745,  .478}0.64 & \cellcolor[rgb]{ .384,  .745,  .478}0.64 & \cellcolor[rgb]{ .988,  1,  .992}0.69 & \cellcolor[rgb]{ .384,  .745,  .478}0.64 & \cellcolor[rgb]{ .933,  .976,  .945}0.68 \\
\hline
BB20037 & \cellcolor[rgb]{ .384,  .745,  .478}0.13 & \cellcolor[rgb]{ .549,  .816,  .62}0.16 & \cellcolor[rgb]{ .8,  .922,  .831}0.21 & \cellcolor[rgb]{ .886,  .957,  .906}0.23 & \cellcolor[rgb]{ .988,  1,  .992}0.25 \\
\hline
BB30003 & \cellcolor[rgb]{ .988,  1,  .992}0.30 & \cellcolor[rgb]{ .384,  .745,  .478}0.28 & \cellcolor[rgb]{ .706,  .878,  .749}0.29 & \cellcolor[rgb]{ .584,  .827,  .647}0.29 & \cellcolor[rgb]{ .945,  .98,  .957}0.30 \\
\hline
BB30021 & \cellcolor[rgb]{ .941,  .98,  .953}0.36 & \cellcolor[rgb]{ .384,  .745,  .478}0.32 & \cellcolor[rgb]{ .988,  1,  .992}0.36 & \cellcolor[rgb]{ .384,  .745,  .478}0.32 & \cellcolor[rgb]{ .784,  .914,  .82}0.35 \\
\hline
BB30026 & \cellcolor[rgb]{ .859,  .945,  .882}0.19 & \cellcolor[rgb]{ .384,  .745,  .478}0.15 & \cellcolor[rgb]{ .922,  .973,  .937}0.20 & \cellcolor[rgb]{ .702,  .878,  .745}0.18 & \cellcolor[rgb]{ .988,  1,  .992}0.20 \\
\hline
BB30011 & \cellcolor[rgb]{ .988,  1,  .992}0.35 & \cellcolor[rgb]{ .384,  .745,  .478}0.32 & \cellcolor[rgb]{ .804,  .922,  .835}0.35 & \cellcolor[rgb]{ .988,  1,  .992}0.35 & \cellcolor[rgb]{ .988,  1,  .992}0.35 \\
\hline
BB40009 & \cellcolor[rgb]{ .384,  .745,  .478}0.19 & \cellcolor[rgb]{ .384,  .745,  .478}0.19 & \cellcolor[rgb]{ .533,  .808,  .604}0.20 & \cellcolor[rgb]{ .384,  .745,  .478}0.19 & \cellcolor[rgb]{ .988,  1,  .992}0.24 \\
\hline
BB40019 & \cellcolor[rgb]{ .988,  1,  .992}0.43 & \cellcolor[rgb]{ .384,  .745,  .478}0.29 & \cellcolor[rgb]{ .745,  .898,  .784}0.37 & \cellcolor[rgb]{ .384,  .745,  .478}0.29 & \cellcolor[rgb]{ .624,  .847,  .682}0.34 \\
\hline
BB40033 & \cellcolor[rgb]{ .988,  1,  .992}0.13 & \cellcolor[rgb]{ .384,  .745,  .478}0.06 & \cellcolor[rgb]{ .867,  .949,  .886}0.11 & \cellcolor[rgb]{ .384,  .745,  .478}0.06 & \cellcolor[rgb]{ .867,  .949,  .886}0.11 \\
\hline
BB40006 & \cellcolor[rgb]{ .384,  .745,  .478}0.00 & \cellcolor[rgb]{ .384,  .745,  .478}0.00 & \cellcolor[rgb]{ .886,  .957,  .906}0.09 & \cellcolor[rgb]{ .886,  .957,  .906}0.09 & \cellcolor[rgb]{ .988,  1,  .992}0.11 \\
\hline
BB50002 & \cellcolor[rgb]{ .718,  .886,  .761}0.50 & \cellcolor[rgb]{ .384,  .745,  .478}0.40 & \cellcolor[rgb]{ .851,  .941,  .875}0.54 & \cellcolor[rgb]{ .718,  .886,  .761}0.50 & \cellcolor[rgb]{ .988,  1,  .992}0.58 \\
\hline
BB50009 & \cellcolor[rgb]{ .988,  1,  .992}0.24 & \cellcolor[rgb]{ .835,  .933,  .863}0.20 & \cellcolor[rgb]{ .957,  .984,  .965}0.23 & \cellcolor[rgb]{ .384,  .745,  .478}0.08 & \cellcolor[rgb]{ .835,  .933,  .863}0.20 \\
\hline
BB50014 & \cellcolor[rgb]{ .384,  .745,  .478}0.41 & \cellcolor[rgb]{ .384,  .745,  .478}0.41 & \cellcolor[rgb]{ .867,  .949,  .886}0.44 & \cellcolor[rgb]{ .988,  1,  .992}0.44 & \cellcolor[rgb]{ .988,  1,  .992}0.44 \\
\hline
BB50006 & \cellcolor[rgb]{ .749,  .898,  .788}0.21 & \cellcolor[rgb]{ .384,  .745,  .478}0.14 & \cellcolor[rgb]{ .839,  .937,  .867}0.23 & \cellcolor[rgb]{ .655,  .859,  .71}0.19 & \cellcolor[rgb]{ .988,  1,  .992}0.26 \\
\hline
\end{tabular}\label{tab:pmao-5pw-b}\end{table}

\begin{table*}[!htbp]
\small
\caption{\underline{Friedman Aligned Ranks test (Column 2):} Friedman Aligned ranks (lower is better) of PASTA and the two PMAO variants based on the best FN rates reported in Table~\ref{tab:pmao-5pw-b}. We also show the computed statistics and corresponding $ p $-value.
\underline{Holm's post-hoc procedure (Columns 3 - 5):} Comparison among PASTA and two PMAO variants using the Holm's post-hoc procedures. Each entry shows the adjusted $p$-value which indicates the significance of the difference in performance between two methods.}
%	\begin{adjustwidth}{-0.3cm}{}
\begin{tabular}{|l|r||ccc|}
\hline
\multicolumn{1}{|c|}{1} & \multicolumn{1}{c||}{2} & \multicolumn{1}{c|}{3} & \multicolumn{1}{c|}{4} & 5 \\
\hline
\multirow{2}{*}{\makecell{Method}} & \multirow{2}{*}{\makecell{Friedman\\Aligned Rank*}} & \multicolumn{3}{c|}{Holm's adjusted $p$-value} \\
\cline{3-5}          &       & \multicolumn{1}{l|}{PMAO-5DW} & \multicolumn{1}{l|}{PMAO-5RW} & \multicolumn{1}{l|}{PASTA} \\
\hline
PMAO-5DW & 27.0000 & \multicolumn{1}{c|}{-} & \multicolumn{1}{r|}{0.1972} & \multicolumn{1}{r|}{\cellcolor[rgb]{ .384,  .745,  .478}0.0018} \\
\hline
PMAO-5RW & 34.7917 & \multicolumn{1}{r|}{0.1972} & \multicolumn{1}{c|}{-} & \multicolumn{1}{r|}{0.0650} \\
\hline
PASTA & 47.7083 & \multicolumn{1}{r|}{\cellcolor[rgb]{ .384,  .745,  .478}0.0018} & \multicolumn{1}{r|}{0.0650} & - \\
\hline
*Statistic & 8.5117 & \multicolumn{3}{c|}{\multirow{2}{*}{N/A}} \\
\cline{1-2}    *$p$-value & 0.0142 & \multicolumn{3}{c|}{} \\
\hline
\end{tabular}\label{tab:test-ml-weights}
%\end{adjustwidth}
\end{table*}
\end{comment}


